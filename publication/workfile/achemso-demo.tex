
%%%%%%%%%%%%%%%%%%%%%%%%%%%%%%%%%%%%%%%%%%%%%%%%%%%%%%%%%%%%%%%%%%%%%
%% This is a (brief) model paper using the achemso class
%% The document class accepts keyval options, which should include
%% the target journal and optionally the manuscript type.
%%%%%%%%%%%%%%%%%%%%%%%%%%%%%%%%%%%%%%%%%%%%%%%%%%%%%%%%%%%%%%%%%%%%%
\documentclass[journal=jacsat,manuscript=article]{achemso}

%%%%%%%%%%%%%%%%%%%%%%%%%%%%%%%%%%%%%%%%%%%%%%%%%%%%%%%%%%%%%%%%%%%%%
%% Place any additional packages needed here.  Only include packages
%% which are essential, to avoid problems later. Do NOT use any
%% packages which require e-TeX (for example etoolbox): the e-TeX
%% extensions are not currently available on the ACS conversion
%% servers.
%%%%%%%%%%%%%%%%%%%%%%%%%%%%%%%%%%%%%%%%%%%%%%%%%%%%%%%%%%%%%%%%%%%%%
\usepackage[version=3]{mhchem} % Formula subscripts using \ce{}

%%%%%%%%%%%%%%%%%%%%%%%%%%%%%%%%%%%%%%%%%%%%%%%%%%%%%%%%%%%%%%%%%%%%%
%% If issues arise when submitting your manuscript, you may want to
%% un-comment the next line.  This provides information on the
%% version of every file you have used.
%%%%%%%%%%%%%%%%%%%%%%%%%%%%%%%%%%%%%%%%%%%%%%%%%%%%%%%%%%%%%%%%%%%%%
%%\listfiles

%%%%%%%%%%%%%%%%%%%%%%%%%%%%%%%%%%%%%%%%%%%%%%%%%%%%%%%%%%%%%%%%%%%%%
%% Place any additional macros here.  Please use \newcommand* where
%% possible, and avoid layout-changing macros (which are not used
%% when typesetting).
%%%%%%%%%%%%%%%%%%%%%%%%%%%%%%%%%%%%%%%%%%%%%%%%%%%%%%%%%%%%%%%%%%%%%
\newcommand*\mycommand[1]{\texttt{\emph{#1}}}
\newcommand{\mylanguage}{american,ngerman}
\newcommand{\mybiblatexdashed}{false}  %% "true" or "false"
%% If true: replace recurring reference authors with a dash.

\newcommand{\mybiblatexbackref}{true}  %% "true" or "false"
%% If true: create backward links from reference to citations.

\newcommand{\mybiblatexfile}{references-biblatex.bib}


\usepackage[utf8]{inputenc} 
%%%%%%%%%%%%%%%%%%%%%%%%%%%%%%%%%%%%%%%%%%%%%%%%%%%%%%%%%%%%%%%%%%%%%
%% Meta-data block
%% ---------------
%% Each author should be given as a separate \author command.
%%
%% Corresponding authors should have an e-mail given after the author
%% name as an \email command. Phone and fax numbers can be given
%% using \phone and \fax, respectively; this information is optional.
%%
%% The affiliation of authors is given after the authors; each
%% \affiliation command applies to all preceding authors not already
%% assigned an affiliation.
%%
%% The affiliation takes an option argument for the short name.  This
%% will typically be something like "University of Somewhere".
%%
%% The \altaffiliation macro should be used for new address, etc.
%% On the other hand, \alsoaffiliation is used on a per author basis
%% when authors are associated with multiple institutions.
%%%%%%%%%%%%%%%%%%%%%%%%%%%%%%%%%%%%%%%%%%%%%%%%%%%%%%%%%%%%%%%%%%%%%
\author{Florian Kluibenschedl}
\altaffiliation{A shared footnote}
\author{Mathias Scherl}
\altaffiliation{Adresse: Bahnhofstraße 10, 6410 Telfs}
\author{Dr. Thomas Müller}
\altaffiliation{A shared footnote}
\email{i.k.groupleader@unknown.uu}
\phone{+123 (0)123 4445556}
\fax{+123 (0)123 4445557}
\affiliation[Unknown University]
{Department of Chemistry, Unknown University, Unknown Town}
\alsoaffiliation[Second University]
{Department of Chemistry, Second University, Nearby Town}
\author{Susanne K. Laborator}
\email{s.k.laborator@bigpharma.co}
\affiliation[BigPharma]
{Lead Discovery, BigPharma, Big Town, USA}
\author{Kay T. Finally}
\affiliation[Unknown University]
{Department of Chemistry, Unknown University, Unknown Town}
\alsoaffiliation[Second University]
{Department of Chemistry, Second University, Nearby Town}



\usepackage{biblatex}  %% remove, if using BibTeX instead of biblatex

%\addbibresource{references-biblatex}

%%%%%%%%%%%%%%%%%%%%%%%%%%%%%%%%%%%%%%%%%%%%%%%%%%%%%%%%%%%%%%%%%%%%%
%% The document title should be given as usual. Some journals require
%% a running title from the author: this should be supplied as an
%% optional argument to \title.
%%%%%%%%%%%%%%%%%%%%%%%%%%%%%%%%%%%%%%%%%%%%%%%%%%%%%%%%%%%%%%%%%%%%%
\title[VWA im Bereich Chemie]
  {Direkte Analyse von Chlorophyllkataboliten}

%%%%%%%%%%%%%%%%%%%%%%%%%%%%%%%%%%%%%%%%%%%%%%%%%%%%%%%%%%%%%%%%%%%%%
%% Some journals require a list of abbreviations or keywords to be
%% supplied. These should be set up here, and will be printed after
%% the title and author information, if needed.
%%%%%%%%%%%%%%%%%%%%%%%%%%%%%%%%%%%%%%%%%%%%%%%%%%%%%%%%%%%%%%%%%%%%%
\abbreviations{IR,NMR,UV}
\keywords{American Chemical Society, \LaTeX}

\begin{document}
%%%%%%%%%%%%%%%%%%%%%%%%%%%%%%%%%%%%%%%%%%%%%%%%%%%%%%%%%%%%%%%%%%%%%
%% The manuscript does not need to include \maketitle, which is
%% executed automatically.  The document should begin with an
%% abstract, if appropriate.  If one is given and should not be, the
%% contents will be gobbled.
%%%%%%%%%%%%%%%%%%%%%%%%%%%%%%%%%%%%%%%%%%%%%%%%%%%%%%%%%%%%%%%%%%%%%
\begin{abstract}
  Im Rahmen eines Praktikums am Organischen Institut der Universität Innsbruck wurden die Chlorophyll-Kataboliten frisch gepflückter Brokkoliblätter einer direkten Analyse mit MS-Leafspray unterzogen. Chlorophyll-Kataboliten sind die Endprodukte des Abbauprozesses von Chlorophyll.

MS-Leafspray stellt dabei eine neue, moderne Methode der Massenspektrometrie dar, die es ermöglicht, Probenmaterial in natürlicher Umgebung zu analysieren. Nach einer Erstidentifikation über MS-Leafspray wurde das Ergebnis mit LC-MS verifiziert. Die mit beiden Methoden gefundenen Chlorophyll-Kataboliten lauten wie folgt: \textit{Bo}-NCC-1, \textit{Bo}-NCC-3, \textit{Bo}-DNCC, \textit{Bo}-DNCC-2, \textit{Bo}-DYCC und \textit{Bo}-YCC. Im Vergleich zur Brokkolifrucht konnten vier weitere Chlorophyll-Kataboliten gefunden werden, was neue Fragen in Bezug auf das Verständnis des Abbauprozesses aufwirft. 

Die Reaktionsprodukte einer Reaktion am Blatt mit Essigsäureanhydrid konnten mit MS-Leafspray nachgewiesen werden. Auf Basis diverser Fragmentierungen wird vorgeschlagen, dass die Reaktion nur an einer der beiden freien Carbonsäuren der Chlorophyll-Kataboliten erfolgt. 

Im Rahmen der massenspektrometrischen Analysen wurden Fragmentierungsdiagramme erstellt, von denen angesichts der Ergebnisse vermutet wird, dass sie charakteristisch für bestimmte Chlorophyll-Kataboliten sind. Eine Interpretationsmöglichkeit der Diagramme wurde vorgeschlagen. 
\end{abstract}

%%%%%%%%%%%%%%%%%%%%%%%%%%%%%%%%%%%%%%%%%%%%%%%%%%%%%%%%%%%%%%%%%%%%%
%% Start the main part of the manuscript here.
%%%%%%%%%%%%%%%%%%%%%%%%%%%%%%%%%%%%%%%%%%%%%%%%%%%%%%%%%%%%%%%%%%%%%
\section{Introduction}
This is a paragraph of text to fill the introduction of the
demonstration file.  The demonstration file attempts to show the
modifications of the standard \LaTeX\ macros that are implemented by
the \textsf{achemso} class.  These are mainly concerned with content,
as opposed to appearance.\citep{MassSpectrometry}

\section{Results and discussion}



%%%%%%%%%%%%%%%%%%%%%%%%%%%%%%%%%%%%%%%%%%%%%%%%%%%%%%%%%%%%%%%%%%%%%
%% The appropriate \bibliography command should be placed here.
%% Notice that the class file automatically sets \bibliographystyle
%% and also names the section correctly.
%%%%%%%%%%%%%%%%%%%%%%%%%%%%%%%%%%%%%%%%%%%%%%%%%%%%%%%%%%%%%%%%%%%%%
\bibliography{references-biblatex}


%%%%%%%%%%%%%%%%%%%%%%%%%%%%%%%%%%%%%%%%%%%%%%%%%%%%%%%%%%%%%%%%%%%%%
%% The "tocentry" environment can be used to create an entry for the
%% graphical table of contents.
%%%%%%%%%%%%%%%%%%%%%%%%%%%%%%%%%%%%%%%%%%%%%%%%%%%%%%%%%%%%%%%%%%%%%


\end{document}
