\chapter{Allgemeine Arbeits- und Analysemethoden}

\section{Herstellung einer Pufferlösung}

Herstellung eines 3.2 mM \ch{NH4Ac} Puffers für eine Analyse mit \gls{hplc}:

Es wurden 0.1542 g Ammoniumacetat eingewogen und in 0.5 L Deionat gelöst.

\section{Probenmaterial}

Als Probe dienten Brokkoliblätter der Brokkolipflanze aus dem Garten meiner Oma (Grundfeld Telfs, Austria). Die Brokkoliblätter wurden im Zeitraum von 03.08.2017 bis 11.09.2017 jeweils in der früh zwischen 07:20 und 07:30 gesammelt. Der Zeitraum für Lagerung und Analyse der Brokkoliblätter betrug maximal drei Tage (danach wurden neue Brokkoliblätter gesammelt). \\

\section{Synthese der Anhydride} \label{sec:ReaktionEssig}

Im Rahmen der direkten Analyse der \gls{Chl-K}en wurden diese einer Reaktion mit Essigsäureanhydrid unterzogen. Die Reaktion wurde dabei direkt am Blatt durchgeführt. Dazu wurden in einem Exikator getrocknete Blätter auf eine Fläche von ca. 2 cm\textsuperscript{2} (entspricht einer Blattmasse von ca. 0.01 g) mit einer Rasierklinge zugeschnitten und anschließend mit 300 \si{\uL} Essigsäureanhydrid beidseitig benetzt. Die 300 \si{\uL} ergaben sich aus experimentellen Erfahrungen durch Vorversuche (zu viel LM löst Teile des Blattes heraus, was zu einer Verringerung der Intensitäten in \gls{hplc} und MS-Leafspray führt). 

Die Reaktion wurde durch eine Vakuumpumpe mit zwischengeschalteter Kühlfalle gestoppt (Verweildauer in der Vakuumpumpe betrug ca. 5 min). Dabei wurde das verbleibende Reagenz abgesaugt. Das Blatt wurde dann entsprechend der folgenden Analysenmethode aufgearbeitet.

Es wurden Reaktionszeiten zwischen 7 min und 6 h ausprobiert. Für die Darstellung und Diskussion der MS-Leafspray Versuche wurden die Reaktionsprodukte nach 22 min Reaktionsdauer verwendet, wohingegen für die Darstellung der \gls{hplc} Versuche die Reaktionsprodukte nach 3 h Reaktionsdauer.

\section{Erstellen von Fragmentierungsdiagrammen} \label{sec:fragmentierungsdiagramme}

Zu den \gls{Chl-K}en wurden Fragmentierungsdiagramme erstellt. Dazu wurden die Intensitäten der einzelnen beobachteten Fragmentierungen im Massenspektrometer (aufgenommen im CID Modus) zur jeweiligen aufgewendeten, \gls{nKE} aufgenommen. 

Auf der Abszisse des erhaltenen Diagramms wird die \gls{nKE} (von 0 bis 100 NKE, gemessen wurde in fünfer Schritten) aufgetragen, auf der Ordinate die Intensität der jeweiligen Fragmentierung, bezogen auf den höchsten Peak (entspricht 100 \%), der während der gesamten Aufnahme beobachtet wurde. 

Die erhaltenen Kurven wurden mit einem Savitzky-Golay Filter \cite{scipy} geglättet (Programmcode Anhang \ref{sec:AnhangFragmentierungsdiagramme}) und werden im folgenden als Fragmentierungsdiagramme bezeichnet. Ein Nachteil bei der Behandlung mit diesem Filter ist, dass in manchen Fällen die Intensitäten der Fragmentierungen bei einer NKE von null nicht gleich null sind. Es wird im Folgenden angenommen, dass dies dennoch so ist. \\

Mit den Fragmentierungsdiagrammen wird versucht, herauszufinden, ob bestimmte Abspaltungen der \gls{Chl-K}en charakteristische Muster aufweisen. Dieses Wissen könnte für eine zukünftige Strukturaufklärung sowie für die wünschenswerte Schnellidentifikation von \gls{Chl-K}en mittels MS-Leafspray von Relevanz sein. Ebenso wird gehofft, eine Verbesserungsmöglichkeit von MS-Leafspray zu finden.