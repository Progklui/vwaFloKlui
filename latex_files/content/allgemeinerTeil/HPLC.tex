\section{HPLC}

Die \gls{hplc} ist eine chromatograpische Methode, um lösliche Stoffe analytisch und präparativ zu trennen. \cite[S. 165]{Chromatographie} 

Das Trennen der Stoffe basiert auf ihren unterschiedlichen chemischen Eigenschaften. Die Stoffe werden gelöst und bilden zusammen mit dem Laufmittel die mobile Phase, die an einer stationären Phase (in der Trennsäule) vorbeiströmt, wobei es zu Wechselwirkungen zwischen den gelösten Stoffen mit der stationären Phase kommt. Aufgrund den unterschiedlichen Affinitäten zu einer Phase hält sich jeder Stoff unterschiedlich lang in der stationären Phase auf. Die Verweildauer eines Stoffes in der Trennsäule wird als Retentionszeit bezeichnet. \cite[S. 31-32]{Chromatographie} Die Retentionszeit wird über Detektoren bestimmt, die die Änderung der Zusammensetzung der mobilen Phase feststellen und das Ergebnis in einem Chromatogramm darstellen. \cite[S. 46]{Chromatographie} 

Für die Experimente wurde die Methode der \gls{rp} Chromatographie angewandt. Dabei ist die mobile Phase polar und die stationäre Phase unpolar (als unpolare Phase dienen beispielsweise Silane mit langen Kohlenwasserstoffketten). \cite[S. 189]{Chromatographie}\\

Im Rahmen meiner Arbeit wurde die \gls{hplc} verwendet, um die Stoffe im seneszenten Blatt zu trennen und entsprechende \gls{Chl-K}en zu isolieren. Die Identifikation der \gls{Chl-K}en erfolgte dabei durch einen UV/Vis Detektor (200 nm - 800 nm) sowie durch ein dazu gekoppeltes Massenspektrometer (=\gls{lcms}). Um die \gls{Chl-K}en mit einem hochauflösenden Massenspektrometer zu fragmentieren, wurden die Verbindungen zu jenen Zeiten, zu denen sie in der \gls{hplc} jeweils eluieren in \gls{eppi} gesammelt.\\

Die Herstellung eines Blattextraktes für die Analyse mit der \gls{hplc} wird in Kapitel \ref{sec:HPLCAufarbeitungderProbe} beschrieben. Aufgrund dieser speziellen Aufarbeitung des Blattes zählt die Methode der \gls{hplc} nicht mehr zur direkten Analyse. Sie wurde lediglich verwendet, um die Ergebnisse von MS-Leafspray zu verifizieren.
