\chapter*{Vorwort und Danksagung}
\label{cha:Vorwort}

\begin{quotation}
\glqq Als ich vierzehn war, war mein Vater so unwissend. Ich konnte den alten Mann kaum in meiner Nähe ertragen. Aber mit einundzwanzig war ich verblüfft, wieviel er in sieben Jahren dazugelernt hatte.\grqq - Mark Twain
\end{quotation}

Im Rückblick auf meine VWA bin ich froh, dieses Forschungsthema gewählt zu haben. Der Wissensgewinn, den ich dabei erfahren konnte ist ungefähr so groß, wie der im Zitat beschriebene\footnote{obwohl ich nicht einundzwanzig bin}. Auch wenn die Auseinandersetzung mit diesem Thema nicht immer leicht war, machte es unheimlich viel Spaß, sich mit Fragestellungen sowohl theoretisch als auch praktisch in Form von Experimenten zu beschäftigen, mit dem Ziel, ein solides Verständnis des Forschungsobjektes zu erlangen.\\

Die eingehendere Beschäftigung mit diesem Thema ermöglichte ein Praktikum im Rahmen des \textit{Sparkling Science} Projektes am Institut für Organische Chemie Innsbruck, das ich im August 2017 absolvieren durfte. Betreut wurde ich dabei von Herrn Dr. Thomas Müller, dem ich an dieser Stelle einen großen Dank für seine Geduld und die vielen Ratschläge aussprechen möchte. Er führte mich in alle für die Analyse von Chlorophyll-Kataboliten notwendigen Arbeiten ein und zeigte mir so zum Beispiel, wie ein Massenspektrometer, eine HPLC oder ein Exikator bedient werden. In der zweiten Woche konnte ich bereits selbständig agieren und die Experimente (fast) ganz nach meinen Wünschen planen und durchführen. 

Die Besprechung und Diskussion der Ergebnisse war selbstverständlich Teil der täglichen Praxis und es machte mir ungeheuer viel Spaß, meine geglaubten Erkenntnisse\footnote{im Gespräch erwies es sich des öfteren, dass ich mich etwas vertan hatte} zu besprechen und mitzuteilen. So konnte ich Einblick in andere Denkweisen gewinnen und bekam von Herrn Müller immer wieder Publikationen zum Lesen, die mir Denkanstöße und Impulse für weitere Analysen gaben. Auf diese Weise wurde ich zum Beispiel auf die Möglichkeit der Erstellung von Fragmentierungsdiagrammen aufmerksam. 

Ich denke, ich werde mich an diese Zeit noch lange zurückerinnern können, da ich dabei viel in puncto Zeitmanagement, Planung und dem generellen Ablauf des Forschungsalltags an sich erfahren konnte. Mein Wunsch, eines Tages nach einem erfolgreich absolviertem Studium der Naturwissenschaften in der Forschung tätig zu sein, wurde in dieser Zeit bestärkt. Mir gefällt einfach der Blick auf die Natur aus einer forschungstechnischen Perspektive heraus, was ich auch versuche, in dieser Arbeit ersichtlich zu machen.\\

Mein weiterer Dank ist an meinen Betreuungslehrer Mag. Mathias Scherl gerichtet. Er schaffte es in der fünften Klasse, mein Interesse an der Chemie zu wecken, dessen Effekt bis heute ungebremst anhält. Durch ihn wurde ich erst auf die Möglichkeit des Praktikums hingewiesen und mein Interesse an der Erforschung des Abbauprozesses von Chlorophyll geweckt. Zusätzlich steht er immer für chemiespezifische Fragen bereit, was ich sehr schätze und durch deren Diskussion ich viel gelernt habe und immer noch lerne. 

Ansonsten möchte ich mich bei allen weiteren beteiligten Personen bedanken, die in irgendeinster Weise bei der Verfassung dieser Arbeit hilfreich waren\footnote{ein großer Dank gilt meiner Familie und Freunden}. Insbesondere seien hier Stefanie Schatz und Theresia Erhart genannt, ebenfalls vom Organischen Institut, die mich während des Praktikums betreuten.  

Zu guter Letzt möchte ich mich bei den Brokkoliblättern aus dem Garten meiner Oma bedanken. Ohne dieses wahrhaftig erstklassige Forschungsobjekt wäre ich unter Umständen zu gar keinem Ergebnis gekommen. Ihr wart ein wahrer Glücksgriff - DANKE! \\

Es hat mir somit das Beschäftigen mit diesem Thema sehr viel Freude bereitet. Das Ergebnis meiner Bemühungen soll in der vorliegenden Arbeit ersichtlich werden. Zu meiner Motivation zum Experimentieren an sich: \\

\begin{quotation}
 \glqq It is well to remember that most arguments in favor of not trying an experiment are too flimsily based.\grqq - R.B. Woodward
 \end{quotation} 

