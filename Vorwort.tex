\chapter*{Vorwort und Danksagung}
\label{cha:Vorwort}

\glqq Als ich vierzehn war, war mein Vater so unwissend. Ich konnte den alten Mann kaum in meiner Nähe ertragen. Aber mit einundzwanzig war ich verblüfft, wieviel er in sieben Jahren dazugelernt hatte.\grqq - Mark Twain

Im Rückblick auf meine VWA bin ich froh, dieses Forschungsthema gewählt zu haben. Der Wissensgewinn, den ich dabei erleben konnte ist ungefähr so groß, wie der im Zitat beschriebene\footnote{ich bin zwar nicht einundzwanzig, aber trotzdem}. Zugegeben, die Beschäftigung mit diesem Thema war nicht immer leicht, es war teils harte Arbeit. Über ein Thema zu schreiben, von dem man im Grunde null Ahnung zu Beginn hat, bringt seine Herausforderungen und teils auch Verwirrung mit sich.

Die eingehendere Beschäftigung mit diesem Thema ermöglichte ein Praktikum im Rahmen des \textit{Sparkling Science} Projektes am Institut für Organische Chemie Innsbruck, das ich im August 2017 absolvieren durfte. Betreut wurde ich dabei von Herrn Dr. Thomas Müller, dem ich an dieser Stelle einen großen Dank für seine Geduld und die vielen Ratschläge aussprechen möchte. Er führte mich in die praktischen Arbeiten ein und zeigte mir, wie ein Massenspektrometer bedient wird. In der zweiten Woche konnte ich bereits fast selbständig operieren und die Experimente (fast) ganz nach meinen Wünschen durchführen. Die Besprechung und Diskussion der Ergebnisse war selbstverständlich ebenso ein Teil der alltäglichen Praxis. Ich denke, ich werde mich an diese Zeit noch lange erinnern können, da ich dabei viel in puncto Zeitmanagement und dem Forschungsalltag an sich erfahren konnte. Mein Wunsch, eines Tages nach einem erfolgreich absolviertem Studium der Naturwissenschaften in der Forschung tätig zu sein, wurde bestärkt.

Mein weiterer Dank ist an meinen Betreuungslehrer Mag. Mathias Scherl gerichtet. Er ist unter Umständen hauptverantwortlich für mein Interesse an der Chemie, das seit der 5ten Klasse ohne Abbruch anhält. Durch ihn wurde ich erst auf die Möglichkeit des Praktikums hingewiesen und mein Interesse an der Erforschung des Abbauprozesses von Chlorophyll geweckt. Auch steht er immer für chemiespezifische Fragen bereit, was ich sehr schätze und wodurch ich viel gelernt habe.

Ansonsten möchte ich mich bei allen weiteren beteiligten Personen bedanken, die in irgendeinster Weise bei der Verfassung dieser Arbeit von Hilfe waren\footnote{eine Auflistung aller Personen, die ich bisher getroffen habe, wäre sinnlos, weswegen ich mich hierbei auf die wesentlichen beziehe - Familie und Co.} . 

Zu guter Letzt möchte ich bei den Brokkoliblättern aus dem Garten meiner Oma bedanken. Ohne dieses wahrhaftig erstklassige Forschungsobjekt wäre ich unter Umständen zu gar keinem Ergebnis gekommen. Ihr wart ein wahrer Glücksgriff - DANKE!


