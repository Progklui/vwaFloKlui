\chapter{Experimente LC-MS} 

Die Analyse mit \gls{lcms} diente dazu, die Ergebnisse von MS Leafspray zu überprüfen. In Kombination mit einem hochauflösenden Massenspektrometer wurde zudem die Ermittlung der Strukturen der \gls{Chl-K}en erleichtert. Weiters konnten mit MS-Leafspray nicht gefundene \gls{Chl-K}en identifiziert werden. \\

\section{HPLC-Gradient und Gerätebeschreibungen} \label{sec:HPLCGradient} 

Folgendes HPLC-System wurde verwendet: Shimadzu HPLC-System, manueller Probenehmer, DGU-20A5 Online-Entgaser, LC-20AD-Pumpe, CBM-20A Systemcontroller, SPD-M20A-Diodenarray-Detektor, Jasco FP-920 Fluoreszenzdetektor, Rheodyne-Injektionsventil mit 20 \si{\uL}-Schleife und Phenomenex Hyperclone ODS 5 \si{\um} 250 x 4.6 mm i.d. Spalte verbunden mit einem Phenomenex ODS 4 x 3 mm i.d. Vorsäule bei Raumtemperatur mit einer Flussrate von 0.5 \si{\mL\per\minute}. \textit{Ultraviolett/visible} (UV/Vis): Hitachi U-3000 Spektrophotometer, Lösungsmittel: MeOH oder (Online) HPLC Elutionsgemisch. Die Daten wurden mit Shimadzu LC Solution gesammelt und verarbeitet. 

Gradient: Lösungsmittel A: \gls{meoh}, Lösungsmittel B: \ch{NH4Ac}-Puffer (3.2 mM); Lösungsmittelzusammensetzung (A/B) als Funktion der Zeit: 0-5 min: 20/80 bis 20/80; 5-55 min: 20/80 bis 70/30; 55-60 min: 70/30 bis 100/0; 60-70 min: 100/0 bis 100/0; 70-75 min: 100/0 bis 20/80.

Als Massenspektrometer wurde ein Thermo LTQ Orbitrap XL im positiven Ionenmodus verwendet (3-6 kV Sprayspannung, m/z von 300 - 1000). Dieses wurde auch für die Messungen von Kapitel \ref{sec:ChlKatabolitenESIMS} verwendet.

\section{Aufarbeitung der Probe} \label{sec:HPLCAufarbeitungderProbe}

Um ein Blattextrakt zu erhalten, wurde ein Brokkoliblatt auf eine Größe von ca. 2 \si{cm^{2}} mit einer Rasierklinge zugeschnitten, mithilfe von Mörser und Pistill aufgerieben und mit 2-5 mL \gls{meoh} vermengt (Anm.: hohe Intensitäten in der \gls{hplc} werden durch eine möglichst hohe Konzentration erreicht). Die Lösung wurde für 2 min bei 3000 rpm abzentrifugiert und anschließend mit Wasser im Verhältnis 20:80 verdünnt und nach kurzem Homogenisieren für 7 min (3000 rpm) abzentrifugiert. Von der erhaltenen Lösung wurden 50 \si{\uL} in die 20 \si{\uL} Schleife der \gls{hplc} injiziert. 

Beim Injizieren wurde versucht, die ungelösten Bestandteile im Blattextrakt nicht mitzunehmen, da diese die \gls{hplc} mit der Zeit verunreinigen könnten. Eine Filterung des Blattextraktes erwies sich als Intensitätsverringernd.

Bevor die Blätter wie oben beschrieben aufgerieben wurden, wurden sie in einem Exikator getrocknet (ca. einen halben bis ganzen Tag). Die Blätter, an denen die Reaktion (Kapitel \ref{sec:ReaktionEssig}) stattgefunden hatte, wurden nach dem Stoppen der Reaktion der gleichen Aufarbeitungsmethode unterzogen. 

\section{Theoretische Grundlagen der Auswertung von Online-UV/Vis Spektren} \label{sec:IdentifikationUVVis}

Mithilfe einer HPLC kann bestimmt werden, ob es sich bei einem bestimmten \gls{Chl-K}en um einen \gls{NCC}, \gls{DNCC} oder \gls{YCC} handelt. Man erhält zu jedem Peak im \gls{hplc} Chromatogramm ein Online-UV/Vis Spektrum, das von einem an die \gls{hplc} angeschlossenen UV/Vis Detektor gemessen wurde (Kapitel \ref{sec:HPLCGradient}). \\

Ein \gls{NCC} kann über eine charakteristische Bande bei 315 nm eindeutig bestimmt werden. Die Bande geht dabei auf das konjugierte System von Ring A zurück. Ein \gls{DNCC} besitzt aufgrund seiner deformylierten Carbonylgruppe dieses konjugierte System nicht mehr, weswegen die Bande bei 315 nm verschwindet, der sonstige für einen \gls{NCC} typische Kurvenverlauf bleibt jedoch erhalten. Bei einem \gls{YCC} führt die Reduktion der Verbindung zwischen Ring C und D (Einführung einer Doppelbindung zwischen Position C-15 und C-16) zu einer Erweiterung des konjugierten Systems (nun bestehend aus Ring C und D) und damit zu einer Bande bei 415 nm. \cite{ErsterKatabolit}\\

Alle folgenden UV/Vis Spektren und \gls{hplc} Chromatogramme wurden bei 254 nm detektiert.

\section{Theoretische Grundlagen der Auswertung der MS Spektren} \label{sec:IdentifikationMS}

̈Über das an die HPLC gekoppelte Massenspektrometer wird zu jedem Zeitpunkt eines \gls{hplc} Laufes ein Massenspektrum aufgenommen und ein entsprechendes Chromatogramm erzeugt. Die hervorgehobenen Peaks in den folgenden Chromatogrammen zeigen an, zu welchem Zeitpunkt welcher Chl-Katabolit in Bezug auf seine Molekülmasse gefunden wurde. Da das Massenspektrometer erst nach 10 min an die HPLC gekoppelt wurde, muss man, um die entsprechende Retentionszeit im HPLC Chromatogramm zu erhalten, zu jedem Zeitpunkt im Chromatogramm des Massenspektrometers ca. 11 min dazuzählen (eine Minute wird für das Zurücklegen des Weges von der HPLC zum Massenspektrometer benötigt).  \\

Man kann somit den über Online-UV/Vis Spektren identifizierten \gls{Chl-K}en (Kapitel \ref{sec:IdentifikationUVVis}) eine Molekülmasse zuordnen. Im Rahmen meiner Vorwissenschaftlichen Arbeit erwies es sich jedoch als schwierig, die Resultate der \gls{hplc} mit jenen des Massenspektrometers im Rahmen eines \gls{lcms} Versuches in Einklang zu bringen (die Retentionszeiten der über \gls{hplc} identifizierten \gls{Chl-K}en stimmten mit jenen des Massenspektrometers oft nicht überein). Da die Verwendung von Daten aus der \gls{hplc} zur Analyse der \gls{Chl-K}en nicht das primäre Ziel meiner Arbeit war, spielt dies keine wesentliche Rolle. Aus Gründen der wissenschaftlichen Vollständigkeit, werden die Daten der \gls{hplc} trotzdem präsentiert. Ebenso wird versucht, die Probleme, die sich aus den Daten ergeben, darzustellen.\\

Der Typ des \gls{Chl-K}en wurde somit, sofern möglich durch ein Online-UV/Vis Spektrum bestimmt und mit der vom Massenspektrometer erhaltenen Summenformel und den sich daraus ergebenden strukturellen Möglichkeiten überprüft. War die Zuordnung anhand UV/Vis Spektren aufgrund von Unklarheiten nicht möglich, wurde zur Strukturbestimmung auf die Daten des Massenspektrometers zurückgegriffen.\\

\section{Chlorophyllkataboliten des Brokkoliblattes mithilfe von LC-MS identifiziert}

\subsection{Auswertung der Chromatogramme}

Mithilfe einer \gls{hplc} kann bestimmt werden, ob es sich bei einem bestimmten \gls{Chl-K} um einen NCC, DNCC, YCC oder DYCC handelt. Das Chromatogramm in Abbildung \ref{fig:HPLCChromatogramm} zeigt, welche der Kataboliten mithilfe ihrer UV/Vis Spektren identifiziert werden konnten. Es dürfte sich dabei ob ihrer etwas höheren Intensitäten um die Hauptkataboliten des Brokkoliblattes handeln. Dies müsste jedoch über gezielte quantitative Messungen weiter und genauer untersucht werden.

\begin{figure}[!htbp]
  
  \includegraphics[width=\textwidth]{figures/Kapitel6/keineReaktion/VWA_HPLC_Chromatogramm_keineReaktion.png}
  \caption[HPLC Chromatogramm vor der Reaktion, Quelle: Author]{\gls{hplc} Chromatogramm - die hervorgehobenen Peaks entsprechen den Retentionszeiten und der Art des \gls{Chl-K}, die über ein UV/Vis Spektrum, aufgenommen im online-Modus, bestimmt wurde; gefunden wurden ein \gls{NCC} bei 27.10min. (Abbildung \ref{fig:}), ein DNCC bei 29.75min. (Abbildung \ref{fig:}) und ein YCC bei 30.94min. (Abbildung \ref{fig:})}
  \label{fig:HPLCChromatogramm}
\end{figure}

Über das an die \gls{hplc} gekoppelte Massenspektrometer wurde ebenfalls ein Chromatogramm erzeugt (Abbildung \ref{fig:LCMSChromatogramm}). Die hervorgehobenen Peaks zeigen an, zu welchem Zeitpunkt welcher Katabolit in Bezug auf seine Molekülmasse gefunden wurde. Da das Massenspektrometer erst nach 10min. an die \gls{hplc} gekoppelt wurde, muss man, um die entsprechende Retentionszeit im \gls{hplc} Chromatogramm zu erhalten, zu jedem Zeitpunkt im Chromatogramm des Massenspektrometers (MS Chromatogramm - Abbildung \ref{fig:LCMSChromatogramm}; HPLC Chromatogramm - Abbildung \ref{fig:HPLCChromatogramm}) 10min. dazuzählen. 

\begin{figure}[!htbp]
  \centering
  \includegraphics[width=1\textwidth]{figures/Kapitel6/keineReaktion/Kuerbis_Analyse_keineReaktion2_Ganzes_Spektrum.png}
  \caption[LC-MS Chromatogramm vor der Reaktion, Quelle: Author]{\gls{lcms} Chromatogramm}
  \label{fig:LCMSChromatogramm}
\end{figure}

Es wurden somit mit dem Massenspektrometer die in Tabelle \ref{tab:LCMSKataboliten} aufgelisteten Phyllobiline identifiziert. In dieser Tabelle werden neben den Summenformeln auch die exakten Massen, die Art des \gls{Chl-K} (NCC, DNCC, YCC, DYCC), die Retentionszeit in der \gls{hplc} (soweit eindeutig feststellbar) angegeben. 

Eine so große Anzahl an \gls{Chl-K} wie in Tabelle \ref{tab:LCMSKataboliten} vorzufinden wäre ungewöhnlich. Bei einer Betrachtung der Summenformeln und exakten Molekülmassen fällt jedoch auf, dass sich einige \gls{Chl-K} um genau ein C-Atom und zwei H-Atome unterscheiden (entsprich 14Da). Da alle identifizierten Kataboliten eine freie Carbonsäuregruppe an Position .. besitzen, wird angenommen, dass diese bei der Aufarbeitung der Probe mit \gls{meoh} (Kapitel \ref{sec:HPLCAufarbeitungderProbe}) mit diesem reagieren und einen Methylester ausbilden. In der Spalte Herkunft (abgekürzt mit H.) der Tabelle \ref{tab:LCMSKataboliten} wird demnach festgehalten, von welchem \gls{Chl-K} die jeweilige Verbindung stammt. \\

\begin{table*}\centering
  \ra{1.3}

  \begin{tabular}{cccccc}\toprule
 Bezeichnung & Summenformel & M (in Da) & Typ & RT\textsubscript{HPLC} (in min.) & H. \\
\midrule
\rowcolor{black!20} Bo-DYCC & \ch{C33H37O8N4} & 617.2635 & DYCC & 30.94? & - \\
 Bo-DNCC & \ch{C33H39O8N4} & 619.2793 & DNCC & 26.72 & - \\ 
\rowcolor{black!20} • & \ch{C34H37O8N4} & 629.2639 & • & - & - \\ 
 - & \ch{C34H39O8N4} & 631.2795 & DYCC & 29.91, 30.94 & Bo-DYCC \\ 
\rowcolor{black!20} - & \ch{C34H41O8N4} & 633.2955 & DNCC & - & Bo-DNCC \\ 
 • & \ch{C36H33O7N4} & 633.2339 & • & • & - \\ 
\rowcolor{black!20} Bo-YCC & \ch{C34H37O9N4} & 645.2593 & YCC & - & - \\ 
 Bo-NCC-3 & \ch{C34H39O9N4} & 647.2748 & NCC & 33.04 & - \\ 
\rowcolor{black!20} - & \ch{C35H39O9N4} & 659.2741 & YCC & - & Bo-YCC \\
 Bo-DNCC-2 & \ch{C39H47O13N4} & 779.3181 & DNCC & • & - \\ 
\rowcolor{black!20}Bo-NCC-1 & \ch{C40H49O13N4} & 793.3336 & NCC & 29.91 & - \\ 
 - & \ch{C41H51O13N4} & 807.3491 & NCC & - & Bo-NCC-1 \\ 
\bottomrule
  \end{tabular}
  \caption[Übersicht über die Chl-Kataboliten des Brokkoliblattes, Quelle: Author]{Übersicht über die gefundenen Chl-Kataboliten des Brokkoliblattes und ihren Methylestern, die sich aus der Reaktion der freien Carbonsäure mit \gls{meoh} ergeben (die Summenformeln und die exakten Molekülmassen beziehen sich auf die [M+H]\textsuperscript{+} Ionen)}
  \label{tab:LCMSKataboliten}
\end{table*}

Im Rahmen meiner Vorwissenschaftlichen Arbeit erwies es sich als schwierig, die Resultate der \gls{hplc} mit denen des Massenspektrometers im Rahmen eines \gls{lcms} Versuches in Einklang zu bringen. Da die Verwendung von Daten aus der \gls{hplc} zur Analyse der \gls{Chl-K} nicht das primäre Ziel meiner Arbeit war, spielt dies auch keine wesentliche Rolle. Aus Gründen der wissenschaftlichen Vollständigkeit, werden die Daten der \gls{hplc} trotzdem präsentiert. Ebenso wird versucht, die Probleme, die sich aus den Daten ergeben, zu erklären.

Der Typ des \gls{Chl-K} wurde, sofern dies möglich war durch ein UV/Vis Spektrum im Onlinemodus bestimmt und mit der vom Massenspektrometer erhaltenen Summenformel und den sich daraus ergebenden strukturellen Möglichkeiten überprüft. War die Zuordnung anhand UV/Vis Spektren aufgrund von Unklarheiten nicht möglich, wurde zur Strukturbestimmung auf die Daten des Massenspektrometers zurückgegriffen.\\

Bei einer Retentionszeit von 27.10min. konnte über UV/Vis ein \gls{NCC} (Abbildung \ref{fig:NCC2725}) identifiziert werden, da er bei einer Wellenlänge von 315nm eine charakteristische Bande aufweist. Der von den Retentionszeiten dazugehörige \gls{Chl-K} im Massenspektrum wäre der Bo-DNCC (mit einer Retentionszeit von 17.5min im Massenspektrometer - Abbildung \ref{fig:LCMSChromatogramm}). Bei diesem handelt es sich jedoch um einen \gls{DNCC}. Es wurde versucht, das unlogische Ergebnis durch Überlagerungen mehrer \gls{Chl-K} zu erklären, was aber nicht möglich war (Abbildung \ref{fig:}). Es bleibt somit das zustandekommen dieses UV/Vis Spektrums ungeklärt. 

Bei einer Retentionszeit von 29.75min. konnte ein UV/Vis Spektrum eines \gls{DNCC}s (Abbildung \ref{fig:DNCC2991} aufgenommen werden. Nach den Retentionszeiten im Massenspektrometer (Abbildung \ref{fig:LCMSChromatogramm}) kann diesem UV/Vis Spektrum der \gls{Chl-K} Bo-NCC-1 zugeordnet werden. Auch der Methylester des Bo-YDNCC ist zu dieser Retentionszeit im Massenspektrometer vorzufinden und trägt damit vermutlich zur Entstehung des Signals bei, was die Verzerrungen bewirken könnte.

Bei einer Retentionszeit von 30.94min. ist das UV/Vis Spektrum charakteristisch für einen \gls{YCC} (Abbildung \ref{fig:DNCC2991}). Im Massenspektrometer wurde zu dieser Retentionszeit der Methylester des Bo-YDNCC gefunden (bei einer Retentionszeit von 18.9min). Auch hier lässt sich keine Verbindung finden, bei der die Retentionszeiten von \gls{hplc} und Massenspektrometer exakt zusammenpassen. Es könnte auch hier wieder zu einer Überlagerung kommen (vielleicht mit dem Bo-YDNCC). Diese Überlagerungen könnten durch Isomere der einzelnen \gls{Chl-K} bedingt sein. \\

\begin{figure}[!htbp]
  \includegraphics[width=\textwidth]{figures/Kapitel6/keineReaktion/Kuerbis_Analyse_keineReaktion2_LC-ESI-MS.png}
  \caption[LC-MS Chromatogramm Aufspaltung, Quelle: Author]{\gls{lcms} Chromatogramm}
  \label{fig:LCMSChromatogrammAufspaltung}
\end{figure} 

Um das Zustandekommen der nicht identifizierbaren UV/Vis Spektren zu erklären wurden Diagramme wie in Abbildung \ref{fig:LCMSChromatogrammAufspaltung} erstellt. Es handelt sich dabei um ein Chromatogramm jedes einzelnen im Massenspektrometer während eines \gls{lcms} Laufes identifizierten \gls{Chl-K}. Die Intensitäten wurden auf den höchsten im Zeitraum vorkommenden Peak skaliert. Bei den gekennzeichneten Peaks handelt es sich um jene, bei denen die jeweilige Verbindung die höchste Intensität im Chromatogramm zeigte. Mithilfe dieser Abbildung sollten etwaige Überlagerungen herausgefunden werden. 

\begin{figure}[!htbp]
  \begin{subfigure}[b]{0.5\textwidth}
    \includegraphics[width=\textwidth]{figures/Kapitel6/keineReaktion/NCC2725.png}
    \caption{}
    \label{fig:NCC2725}
  \end{subfigure}
  \hfill
  \begin{subfigure}[b]{0.5\textwidth}
    \includegraphics[width=\textwidth]{figures/Kapitel6/keineReaktion/DNCC2991.png}
    \caption{}
    \label{fig:DNCC2991}
  \end{subfigure}
  
  \begin{subfigure}[b]{0.5\textwidth}
    \includegraphics[width=\textwidth]{figures/Kapitel6/keineReaktion/YCC3094.png}
    \caption{}
    \label{fig:DNCC2991}
  \end{subfigure}
  \caption[UV/Vis online Spektren mit der Charakteristik eines NCC bei 27.10min., eines DNCC bei 29.75min. sowie eines YCC bei 30.94min., Quelle: Author]{UV/Vis online Spektren: (a) charakteristisch für einen \gls{NCC} - RT = 27.25min., (b) charakteristisch für einen \gls{DNCC} - RT = 29.91min., (c) charakteristisch für einen \gls{YCC} - RT = 30.94min.}
\end{figure}

\section{Identifikation der Reaktionsprodukte mithilfe von LC-MS}

Die Produkte der Reaktion mit Essigsäureanhydrid konnten ebenfalls mithilfe von \gls{lcms} identifiziert werden. In Abbildung \ref{fig:HPLCChromatogrammRP} sind die Reaktionsprodukte, die mittels UV/Vis online Spektren identifiziert wurden, dargestellt. Die dazugehörigen UV/Vis Spektren werden in Abbildung \ref{fig:YCC3398}-e dargestellt. Es handelt sich dabei um die Hauptreaktionsprodukte, die dadurch charakterisiert sind, dass sich ihre Retentionszeit nach hinten verschiebt. Sie dürften somit apolarere Eigenschaften besitzen wie die \gls{Chl-K}, was vermutlich durch den Methylester bedingt ist. 

\begin{figure}[!htbp]
  \includegraphics[width=\textwidth]{figures/Kapitel6/Reaktion3h/HPLC_Chromatogramm.png}
  \caption[HPLC Chromatogramm nach 3h Reaktionsdauer, Quelle: Author]{\gls{hplc} Chromatogramm}
  \label{fig:HPLCChromatogrammRP}
\end{figure}

\begin{figure}[!htbp]
  \includegraphics[width=\textwidth]{figures/Kapitel6/Reaktion3h/Kuerbis_Analyse_Reaktion3h_Ganzes_Spektrum.png}
  \caption[LC-MS Chromatogramm nach 3h Reaktionsdauer, Quelle: Author]{\gls{lcms} Chromatogramm}
  \label{fig:LCMSCChromatogrammRP}
\end{figure}

\begin{figure}[!htbp]
  \begin{subfigure}[b]{0.5\textwidth}
    \includegraphics[width=\textwidth]{figures/Kapitel6/Reaktion3h/YCC3398.png}
    \caption{}
    \label{fig:YCC3398}
  \end{subfigure}
  \hfill
  \begin{subfigure}[b]{0.5\textwidth}
    \includegraphics[width=\textwidth]{figures/Kapitel6/Reaktion3h/DNCC3709.png}
    \caption{}
    \label{fig:DNCC3709}
  \end{subfigure}
  
  \begin{subfigure}[b]{0.5\textwidth}
    \includegraphics[width=\textwidth]{figures/Kapitel6/Reaktion3h/NCC3849.png}
    \caption{}
    \label{fig:NCC3849}
  \end{subfigure}
  \hfill
  \begin{subfigure}[b]{0.5\textwidth}
    \includegraphics[width=\textwidth]{figures/Kapitel6/Reaktion3h/DNCC4003.png}
    \caption{}
    \label{fig:DNCC4003}
  \end{subfigure}
  
  \begin{subfigure}[b]{0.5\textwidth}
    \includegraphics[width=\textwidth]{figures/Kapitel6/Reaktion3h/YCC4728.png}
    \caption{}
    \label{fig:YCC4728}
  \end{subfigure}
  \caption[UV/Vis online Spektren mit der Charakteristik eines YCC bei 33.98min., eines DNCC bei 37.09min. eines NCC bei 38.94min., eines DNCC bei 40.03min. sowie eines YCC bei 47.28min., Quelle: Author]{UV/Vis online Spektren: charakeristisch für (a) \gls{YCC} - RT = 33.98min., (b) \gls{DNCC} - RT = 37.09min., (c) \gls{NCC} - RT = 38.94min., (d) \gls{DNCC} - RT = 40.03min., (e) \gls{YCC} - RT = 47.28min.}
\end{figure}


