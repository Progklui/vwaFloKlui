\section{Identifikation der Reaktionsprodukte} \label{sec:RPMSLeafspray}

Für den Nachweis der Reaktionsprodukte wurde der gleiche Versuchsaufbau wie in Kapitel \ref{sec:Versuchsaufbau} beschrieben verwendet. Das Anhydrid als Reaktionsprodukt konnte durch Verwendung von Acetonitril als \gls{lm} isoliert werden.  

\subsection{Reaktionsprodukt des Bo-DNCC}

Das Reaktionsprodukt des \textit{Bo}-DNCC konnte mit m/z = 699 [M+K]\textsuperscript{+} bestimmt werden (Strukturvorschlag - Abbildung \ref{fig:699MKstructure}). Identifiziert wurde es über die charakteristische Abspaltung von Essigsäure (M = 60 Da) bei m/z = 639 [M - \ch{CH3COOH} + K]\textsuperscript{+}. Ein Mechanismus für die Abspaltung wird in Abbildung \ref{fig:699MKelectronMovement} vorgeschlagen. Dieser Mechanismus ähnelt dem Mechanismus der Abspaltung von \gls{meoh} (z. B. beobachtbar bei einem \textit{Cj}-NCC-1), wie in \cite{StructureElucidation} publiziert.\\

\begin{figure}[!htbp]
  \centering
  \includegraphics[scale=0.6]{figures/Kapitel4/Kataboliten/fragmentation_structures/VWA_Katabolit_699.png}
  \caption[Strukturvorschlag des Reaktionsproduktes von \textit{Bo}-DNCC, Quelle: Autor]{Strukturvorschlag des Reaktionsproduktes mit Summenformel \ch{C35H41N4O9}}
  \label{fig:699MKstructure}
\end{figure}

Es wurden Abspaltungen von \ch{H2O} bei m/z = 681 [M - \ch{H2O} + K]\textsuperscript{+}, von \ch{CH3COOH} bei m/z = 639 [M - \ch{CH3COOH} + K]\textsuperscript{+} und von Ring A und Ring D mit \ch{CO2} bei m/z = 311 [M - (Ring A, Ring D, \ch{CO2}) + K]\textsuperscript{+} beobachtet. 

Zur Identifikation der Reaktionsprodukte wurde die \ch{CH3COOH} Abspaltung aufgrund ihrer Prominenz herangezogen (u. a.
Abbildung \ref{fig:699MKstructurediags2}). Das Fragment bei m/z = 599 [M - (\gls{nAb}) + K]\textsuperscript{+} ist interessant, da eine Abspaltung von 100 Da bei anderen Reaktionsprodukten ebenfalls beobachtet wurde. Die anderen Fragmentierungen in Abbildung \ref{fig:699MKLeafspray} konnten nicht zugeordnet werden. \\

\begin{figure}[!htbp]
  \centering
  \includegraphics[width=\textwidth, height=0.6\textwidth]{figures/Kapitel4/Kataboliten/VWA_MS_LeafSpray_699.png}
  \caption[ESI-MS Spektrum des Reaktionsproduktes von Bo-DNCC, Quelle: Autor]{ESI-MS Spektrum des Reaktionsproduktes mit m/z = 699 [M+K]\textsuperscript{+}}
  \label{fig:699MKLeafspray}
\end{figure}

Diskussion der Abspaltung bei m/z = 599 [M - (\gls{nAb}) + K]\textsuperscript{+}: Die Abspaltung von 100 Da bei m/z = 599 [M - (\gls{nAb}) + K]\textsuperscript{+} erreicht im Fragmentierungsdiagramm lokale Maxima bei 15 \gls{nKE} und 30 \gls{nKE}. Lokale Minima befinden sich bei 17 \gls{nKE} und 40 \gls{nKE}, an jenen Stellen, an der die Abspaltung von \ch{CH3COOH} lokale Maxima aufweist (Abbildung \ref{fig:699MKstructurediags2}). Daraus könnte man Informationen über den Mechanismus der Abspaltung ableiten. Man könnte sagen, dass die Abspaltung von 100 Da einhergeht mit jener von \ch{CH3COOH} und dass sie mechanistisch miteinander verknüpft sind, also, dass bevor einer Abspaltung des Fragments mit 100 Da \ch{CH3COOH} abgespalten werden muss. Es ließe sich damit erklären, warum bei einem Maximum der einen Abspaltung die andere Abspaltung ein Minimum aufweist.\\ 


\begin{figure}[!htbp]
  \begin{subfigure}[b]{0.6\textwidth}
    \includegraphics[width=\textwidth, height=\textwidth]{figures/Kapitel4/Kataboliten/diags/699CID-savgol2.png}
    \caption{}
    \label{fig:699MKLeafspraydiags1}
  \end{subfigure}
  \hfill
  \begin{subfigure}[b]{0.6\textwidth}
    \includegraphics[width=\textwidth, height=\textwidth]{figures/Kapitel4/Kataboliten/diags/699CID-savgol1.png}
    \caption{}
    \label{fig:699MKstructurediags2}
  \end{subfigure}
  
  \caption[Fragmentierungsdiagramme des Reaktionsproduktes von \textit{Bo}-DNCC, Quelle: Autor]{(a) Fragmentierungsdiagramm des \textit{Bo}-DNCC mit allen beobachteten Abspaltungen (blau = 699 [M+K]\textsuperscript{+}, orange = 681 [M - \ch{H2O} + K]\textsuperscript{+}, grün = 663 [M - (2 x \ch{H2O}) + K]\textsuperscript{+}, rot = 643 [M - (\gls{nAb}) + K]\textsuperscript{+}, violett = 639 [M - \ch{CH3COOH} + K]\textsuperscript{+}, braun = 627 [M - (\gls{nAb}) + K]\textsuperscript{+}, pink = 599 [M - (\gls{nAb}) + K]\textsuperscript{+}, grau = 534 [M - (\gls{nAb}) + K]\textsuperscript{+}, hellgrün = 443 [M - (\gls{nAb}) + K]\textsuperscript{+}, türkis = 432 [M - (\gls{nAb}) + K]\textsuperscript{+}), (b) Fragmentierungsdiagramm mit ausgewählten Abspaltungen (blau = 699 [M+K]\textsuperscript{+}, orange = 681 [M - \ch{H2O} + K]\textsuperscript{+}, grün = 639 [M - \ch{CH3COOH} + K], rot = 599 [M - (\gls{nAb}) + K]\textsuperscript{+})}
\end{figure} 

Im Fragmentierungsdiagramm erreicht die \ch{H2O} Abspaltung ein lokales Maximum bei 17 \gls{nKE}. Die Abspaltung nimmt bis zu 30 \gls{nKE} stark ab und bleibt bis zu 90 \gls{nKE} erhalten. Im Vergleich zum Fragmentierungsdiagramm des nicht reagierten \textit{Bo}-DNCC erfolgt die \ch{H2O} Abspaltung bei einer niedrigeren \gls{nKE} und ist länger beobachtbar (vergleiche Abbildungen \ref{fig:619MKLeafspraydiags} und \ref{fig:699MKstructurediags2}). Es gilt zu bedenken, dass beim nicht umgesetzten \textit{Bo}-DNCC das [M+H]\textsuperscript{+}-Ion aufgenommen wurde, wohingegen man beim reagierten \textit{Bo}-DNCC das [M+K]\textsuperscript{+}-Ion analysierte. Der Unterschied im Verlauf der Kurven könnte somit auch durch diesen Umstand bedingt sein.

Die Abspaltung von \ch{CH3COOH} besitzt lokale Maxima bei 20 \gls{nKE} und 45 \gls{nKE}. Das Maximum bei 45 \gls{nKE} ist weniger intensiv. Die Intensität der Abspaltung nimmt dabei kontinuierlich bis zu einer Intensität von 80 \gls{nKE} ab (Abbildung \ref{fig:699MKstructurediags2}). Ein lokales Minimum der Abspaltung befindet sich zwischen 23 \gls{nKE} und 30 \gls{nKE}. \\



\begin{figure}[!htbp]
  \begin{subfigure}[b]{0.5\textwidth}
    \includegraphics[width=\textwidth, height=\textwidth]{figures/Kapitel4/Kataboliten/fragmentation_structures/VWA_Katabolit_699-639_MK_electronMovement.png}
    \caption{}
    \label{fig:699MKelectronMovement}
  \end{subfigure}
  \hfill
  \begin{subfigure}[b]{0.5\textwidth}
    \includegraphics[width=\textwidth, height=\textwidth]{figures/Kapitel4/Kataboliten/fragmentation_structures/VWA_Katabolit_699-639_MK.png}
    \caption{}
    \label{fig:699MK639}
  \end{subfigure}
  \caption[Vorschlag des Mechanismus der \ch{CH3COOH} Abspaltung, Quelle: Autor]{(a) vorgeschlagener Mechanismus der Essigsäureabspaltung und (b) das Produkt, wobei \ch{CH3COOH} als stabiles Neutralteilchen abgespalten wird}
\end{figure}



\pagebreak
\subsection{Reaktionsprodukt von Bo-NCC-3}

Die Masse des Reaktionsproduktes des \textit{Bo}-NCC-3 konnte mit m/z = 727 [M+K]\textsuperscript{+} bestimmt werden. Eine Abspaltung von Essigsäure wurde bei m/z = 667 [M - \ch{CH3COOH} + K]\textsuperscript{+} beobachtet. Weiters wurde eine Abspaltung von \ch{H2O} bei m/z = 709 [M - \ch{H2O} + K]\textsuperscript{+} beobachtet. Bei der Abspaltung bei m/z = 627 [M - (\gls{nAb}) + K]\textsuperscript{+} könnte es sich um die gleiche Abspaltung wie beim Reaktionsprodukt des \textit{Bo}-DNCC handeln, da auch ein Fragment mit M = 100 Da abgespalten wird. Die anderen Abspaltungen (Abbildung \ref{fig:727MKLeafspray}) konnten nicht eindeutig zugeordnet werden.

\begin{figure}[!htbp]
  \centering
  \includegraphics[width=\textwidth, height=0.7\textwidth]{figures/Kapitel4/Kataboliten/VWA_MS_LeafSpray_727.png}
  \caption[ESI-MS des Reaktionsproduktes von Bo-NCC-3, Quelle: Autor]{ESI-MS Spektrum des Reaktionsproduktes bei m/z = 727 [M+K]\textsuperscript{+}}
  \label{fig:727MKLeafspray}
\end{figure}

\begin{figure}[!htbp]
  \centering
  \includegraphics[scale=0.6]{figures/Kapitel4/Kataboliten/fragmentation_structures/VWA_Katabolit_727.png}
  \caption[Strukturvorschlag des Reaktionsproduktes von \textit{Bo}-NCC-3, Quelle: Autor]{Strukturvorschlag des Reaktionsproduktes mit Summenformel \ch{C36H40N4O10}}
  \label{fig:727MKstructure}
\end{figure}

Es wurde beobachtet, dass die Abspaltung von \ch{H2O} bei niedrigeren Energien erfolgt wie jene von \ch{CH3COOH}. Im Vergleich zum Fragmentierungsdiagramm des Reaktionsproduktes des \textit{Bo}-DNCC (Abbildung \ref{fig:699MKLeafspraydiags1}) kann als Charakteristikum der \ch{CH3COOH} Abspaltung ein lokales Maximum bei 45 \gls{nKE} gedeutet werden (Abbildung \ref{fig:699MKstructurediags2} und Abbildung \ref{fig:727MKLeafspraydiags}). Die Abspaltung von \ch{H2O} weist bei beiden Kataboliten ein lokales Maximum bei 15 \gls{nKE} auf und besitzt einen ähnlichen Kurvenverlauf (Abbildung \ref{fig:699MKstructurediags2} und Abbildung \ref{fig:727MKLeafspraydiags}). Dies lässt darauf schließen, dass es sich bei dieser \ch{H2O}-Abspaltung um eine Abspaltung auf ein und derselben Position handelt. Als Position der Abspaltung wird die Hydroxygruppe an Position C-3\textsuperscript{2} des Chl-Kataboliten vorgeschlagen. 

\begin{figure}[!htbp]
  \centering
  \includegraphics[scale=0.7]{figures/Kapitel4/Kataboliten/diags/727CID-savgol.png}
  \caption[Fragmentierungsdiagramm des Reaktionsproduktes von \textit{Bo}-DNCC, Quelle: Autor]{Fragmentierungsdiagramm des Reaktionsproduktes (blau = 727 [M+K]\textsuperscript{+}, orange = 709 [M - \ch{H2O} + K]\textsuperscript{+}, grün = 667 [M - \ch{CH3COOH} + K]\textsuperscript{+}, rot = 691 [M - (\gls{nAb}) + K]\textsuperscript{+}, violett = 627 [M - (\gls{nAb}) + K]\textsuperscript{+}, braun = 562 [M - (\gls{nAb}) + K]\textsuperscript{+}, pink = 472 [M - (\gls{nAb}) + K]\textsuperscript{+})}
  \label{fig:727MKLeafspraydiags}
\end{figure}



\subsection{Reaktionsprodukt von Bo-NCC-1}

Erwartungsgemäß konnte das Reaktionsprodukt des \textit{Bo}-NCC-1 bei m/z = 873 [M+K]\textsuperscript{+} gefunden werden. Es zeigt Abspaltungen von \ch{H2O} bei m/z = 855 [M - \ch{H2O} + K]\textsuperscript{+}, von Essigsäure bei m/z = 813 [M - \ch{CH3COOH} + K]\textsuperscript{+} und von \ch{CH3COOH}, Ring A, Ring D, zweimal \gls{meoh} und \ch{CO} bei m/z = 309 [M - (Ring A, Ring D, 2 x MeOH, \ch{CO})  + K]\textsuperscript{+} (diesselbe Abspaltungssequenz wurde beim Reaktionsprodukt m/z = 661 [M+H]\textsuperscript{+} beobachtet - Kapitel \ref{sec:ESIMSRPBoNCC3}). Beim Fragment m/z = 441 [M - (Ring D, 2 x MeOH, \ch{H2O}) + K]\textsuperscript{+} könnte es sich um eine Abspaltung von Ring D, zweimal \gls{meoh} und \ch{H2O} handeln. 

\begin{figure}[htbp]
  \includegraphics[width=\textwidth, height=0.7\textwidth]{figures/Kapitel4/Kataboliten/VWA_MS_LeafSpray_873.png} 
  \caption[ESI-MS des Reaktionsproduktes von Bo-NCC-1, Quelle: Autor]{ESI-MS des Reaktionsproduktes bei m/z = 873 [M+K]\textsuperscript{+}}
  \label{fig:873MKLeafspray}
\end{figure}

\begin{figure}[htbp]
  \centering
  \includegraphics[scale=0.6]{figures/Kapitel4/Kataboliten/fragmentation_structures/VWA_Katabolit_873.png}
  \caption[Strukturvorschlag des Reaktionsproduktes von Bo-NCC-1, Quelle: Autor]{Strukturvorschlag des Reaktionsproduktes mit Summenformel \ch{C42H50N4O14}}
  \label{fig:873MKstructure}
\end{figure}

Im Fragmentierungsdiagramm sieht man, dass sich das lokale Maximum der Essigsäureabspaltung hin zu niedrigeren Energien verschoben hat. Es befindet sich nun bei  35 \gls{nKE}. Auch die \ch{H2O} Abspaltung verschiebt sich zu niedrigeren Energien und besitzt ein lokales Maximum bei 10 \gls{nKE}. Im Vergleich zum \textit{Bo}-DNCC und \textit{Bo}-NCC-3 nahmen diese Werte um 10 bzw. 5 \gls{nKE} ab. Dieser Zusammenhang wurde in zwei voneinander unabhängigen Experimenten beobachtet (Abbildung \ref{fig:873MKLeafspraydiags1} und Abbildung \ref{fig:873MKstructurediags2}). Die Ursache könnte beim Zuckerring liegen, der die Elektronenverteilung vermutlich so beeinflusst, dass die Abspaltungen bereits bei niedrigeren Energien erfolgen. Eine Analyse möglicher Konformationen wurde aufgrund der zahlreichen Möglichkeiten nicht durchgeführt. Es würde sich jedoch anbieten, eine solche Analyse in zukünftigen Forschungen durchzuführen, um einen besseres Verständnis der Anregung eines Moleküls im Massenspektrometer zu erlangen. \\

Die Abspaltung von Essigsäure bei m/z = 813 [M - \ch{CH3COOH} + K]\textsuperscript{+} wird in den Abbildungen \ref{fig:873MKLeafspraydiags1} und \ref{fig:873MKstructurediags2} fälscherweise mit m/z = 812 angeführt. 

\begin{figure}[!htbp]
  \begin{subfigure}[b]{\textwidth}
    \centering
    \includegraphics[scale=0.9]{figures/Kapitel4/Kataboliten/diags/873CID-savgol1.png}
    \caption{}
    \label{fig:873MKLeafspraydiags1}
  \end{subfigure}
  \hfill
  \begin{subfigure}[b]{\textwidth}
    \centering
    \includegraphics[scale=0.9]{figures/Kapitel4/Kataboliten/diags/873CID-savgol2.png}
    \caption{}
    \label{fig:873MKstructurediags2}
  \end{subfigure}
  
  \caption[Fragmentierungsdiagramm des Reaktionsproduktes von \textit{Bo}-NCC-1, Quelle: Autor]{Fragmentierungsdiagramm des Reaktionsproduktes: (a) Experiment am 13.09.2017 (11:00) - (blau = 873, orange = 855, grün = 837, rot = 812, violett = 765, braun = 708, pink = 594, grau = 534, hellgrün = 441), (b) Experiment am 13.09.2017 (09:45) - schlechter gelungen, weswegen die Abspaltungen nicht so schön wie in Experiment (a) zu sehen sind (blau = 873, orange = 855, grün = 837, rot = 812)}
\end{figure}