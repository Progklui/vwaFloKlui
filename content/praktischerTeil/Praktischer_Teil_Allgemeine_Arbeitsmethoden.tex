\chapter{Allgemeine Arbeits- und Analysemethoden}

\section{Herstellung von Lösungen für eine Analyse mit HPLC}

\section{Fragmentierungsdiagramme} \label{sec:fragmentierungsdiagramme}

Zu jedem Kataboliten wurde ein Fragmentierungsdiagramm erstellt. Dazu werden die Intensitäten der einzelnen beobachteten Fragmentierungen im Massenspektrometer zur aufgewendeten, normaliserten Kollisionsenergie (alle fünf Einheitsschritte) aufgenommen. Auf der Abszisse des erhaltenen Diagramms befindet sich die normalisierte Kollisionsergie in Prozent und auf der Ordinate die Intensität der einzelnen bezogen auf den höchsten Peak, der während der Aufnahme beobachtet wurde, in Prozent. Die erhaltenen Kurven wurden mit einem Savitzky-Golay Filter geglättet (siehe Anhang) und werden im folgenden als Fragmentierungsdiagramme bezeichnet. Ein Nachteil bei der Behandlung mit diesem Filter ist, dass in manchen Fällen die Graphen der Fragmentierungen bei einer normalisierten Kollisionsenergie von null nicht null sind. Es wird im folgenden angenommen, dass dies dennoch so ist.

Die Fragmentierungsdiagramme wurden sowohl im \gls{cid} als auch im \gls{pqd} Modus aufgenommen. 

Es wird damit versucht, herauszufinden, ob bestimmte Abspaltungen der Kataboliten charakteristische Muster aufweisen, um in weiterer Hinsicht, weitere strukturelle Eigenschaften über die Kataboliten mithilfe eines Massenspektrometers zu erfahren. Weiters wird ein Vergleich zwischen den Fragmentierungsdiagrammen von MS Leafspray und des hochauflösenden Massenspektrometers versucht. Für diesen Vergleich wurden nur die Diagramme verwendet, die im \gls{cid} Modus aufgenommen wurden, da das verwendete Massenspektrometer von MS Leafspray nur in diesem Modus operieren konnte.