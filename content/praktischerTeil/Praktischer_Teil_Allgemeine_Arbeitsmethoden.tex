\chapter{Allgemeine Arbeits- und Analysemethoden}

\section{Herstellung von Lösungen für eine Analyse mit HPLC}

\begin{itemize}
\item Herstellung eines 3.2 mM \ch{NH4Ac} Puffers:

Es wurden 0.1542 g Ammoniumacetat abgewogen und in 0.5 L Wasser gelöst.

\end{itemize}

\section{Chemikalien und Material}

\section{Probenmaterial}

Als Probe dienten Brokkoliblätter der Brokkolipflanze aus dem Garten meiner Oma (Grundfeld Telfs, Austria). Die Brokkoliblätter wurden im Zeitraum von 03.08.2017 bis zum 13.09.2017 (die letzten Brokkoliblätter wurden am 11.09.2017 gesammelt) jeweils in der früh zwischen 07:20 und 07:30 gesammelt. Der Zeitraum, in dem die Brokkoliblätter für Analysen verwendet wurden beträgt maximal drei Tage (danach wurden neue Brokkoliblätter gesammelt). \\

\section{Reaktion mit Essigsäureanhydrid} \label{sec:ReaktionEssig}

Im Rahmen der direkten Analyse der \gls{Chl-K}en wurden diese einer Reaktion mit Essigsäureanhydrid unterzogen. Die Reaktion wurde dabei direkt am Blatt durchgeführt. Dazu wurden in einem Exikator getrocknete Blätter auf eine Fläche von \gls{ca} 2cm\textsuperscript{2} (entspricht einer ungefähren Blattmasse von 0.01g) mit einer Rasierklinge zugeschnitten und anschließend mit 300\si{\uL} Essigsäureanhydrid beidseitig benetzt. Die 300\si{\uL} ergaben sich, da das Blatt nicht in der Reagenz schwimmen sollte, da sonst Teile des Blattes aus diesem herausgelöst werden, was zu einer Verringerung der Intensitäten in \gls{hplc} und MS Leafspray führt. 

Die Reaktion wurde durch eine Vakuumpumpe mit zwischengeschalteter Kühlfalle gestoppt (Verweildauer in der Vakuumpumpe betrug \gls{ca} 5min). Dabei wurde die verbleibende Reagenz abgesaugt. Das Blatt wurde dann entsprechend der folgenden Analysenmethode entsprechend aufgearbeitet.

Es wurden Reaktionszeiten zwischen 7min. und 6h ausprobiert. Für die Präsentation und Diskussion der MS Leafspray Versuche wurden die Reaktionsprodukte nach 22min. Reaktionsdauer verwendet, wohingegen für die Präsentation der \gls{hplc} Versuche die Reaktionsprodukte nach 3h Reaktionsdauer.

\section{Erstellen von Fragmentierungsdiagrammen} \label{sec:fragmentierungsdiagramme}

Zu den \gls{Chl-K}en wurden Fragmentierungsdiagramme erstellt. Dazu werden die Intensitäten der einzelnen beobachteten Fragmentierungen im Massenspektrometer (aufgenommen wurde im CID Modus) zur jeweiligen aufgewendeten, \gls{nKE} aufgenommen. 

Auf der Abszisse des erhaltenen Diagramms befindet sich die \gls{nKE} (von 0 bis 100 NKE, gemessen wurde in 5er Schritten) und auf der Ordinate die Intensität der jeweiligen Fragmentierung bezogen auf den höchsten Peak, der während der Aufnahme beobachtet wurde (ebenfalls in Prozent). 

Die erhaltenen Kurven wurden mit einem Savitzky-Golay Filter \cite{scipy} geglättet (Programmcode Anhang) und werden im folgenden als Fragmentierungsdiagramme bezeichnet. Ein Nachteil bei der Behandlung mit diesem Filter ist, dass in manchen Fällen die Intensitäten der Fragmentierungen bei einer NKE von null nicht gleich null sind. Es wird im folgenden angenommen, dass dies dennoch so ist. \\

Mit den Fragmentierungsdiagrammen wird versucht, herauszufinden, ob bestimmte Abspaltungen der Kataboliten charakteristische Muster aufweisen. Dieses Wissen könnte für eine zukünftige Strukturaufklärung sowie für die wünschenswerte Schnellidentifikation von \gls{Chl-K} mittels MS Leafspray von Relevanz sein. Ebenso wird eine Verbesserung von MS Leafspray erhofft.