\chapter{Allgemeine Arbeits- und Analysemethoden}

\section{Herstellung von Lösungen für eine Analyse mit HPLC}

\begin{itemize}
\item Herstellung eines 3.2 mM \ch{NH4Ac} Puffers:

Es wurden 0.1542 g Ammoniumacetat abgewogen und in 0.5 L Wasser gelöst.

\end{itemize}

\section{Chemikalien und Material}

\section{Probenmaterial}

Als Probe dienten Brokkoliblätter der Brokkolipflanze aus dem Garten meiner Oma (Grundfeld Telfs, Austria). Die Brokkoliblätter wurden jeweils in der früh zwischen 07:20 und 07:30 gesammelt. Der Zeitraum, in dem die Brokkoliblätter für Analysen verwendet wurden beträgt maximal drei Tage (danach wurden neue Brokkoliblätter gesammelt). Der Untersuchungszeitraum erstreckt sich von 03.08.2017 bis zum 13.09.2017 (die letzten Brokkoliblätter wurden am 11.09.2017 gesammelt).\\

\section{Reaktion mit Essigsäureanhydrid}

Im Rahmen der direkten Analyse der \gls{Chl-K} wurden diese einer Reaktion mit Essigsäureanhydrid unterzogen. Die Reaktion wurde dabei direkt am Blatt durchgeführt. Dazu wurden in einem Exikator getrocknete Blätter auf eine Fläche von \gls{ca} 2cm\textsuperscript{2} (entspricht einer ungefähren Blattmasse von 0.01g) zugeschnitten und anschließend mit 300\si{\uL} Essigsäureanhydrid beidseitig benetzt. Die 300\si{\uL} ergaben sich, da das Blatt nicht in der Reagenz schwimmen sollte, da sonst Teile des Blattes aus diesem herausgelöst werden, was zu einer Verringerung der Intensitäten in \gls{hplc} und MS-Leafspray führt. 

Die Reaktion wurde durch eine Vakuumpumpe mit zwischengeschalteter Kühlfalle gestoppt (Verweildauer in der Vakuumpumpe betrug \gsl{ca} 5min). Dabei wurde die Reagenz abgesaugt. 

Die Reaktionsdauer betrug zwischen 7min. und 6h. Für die Präsentation und Diskussion der MS-Leafspray Versuche wurden die Reaktionsprodukte nach 22min. Reaktionsdauer verwendet, wohingegen für die Präsentation der \gls{hplc} Versuche die Reaktionsprodukte nach 3h Reaktionsdauer.

\section{Erstellen von Fragmentierungsdiagrammen} \label{sec:fragmentierungsdiagramme}

Zu jedem \gls{Chl-K} wurde ein Fragmentierungsdiagramm erstellt. Dazu werden die Intensitäten der einzelnen beobachteten Fragmentierungen im Massenspektrometer zur jeweiligen aufgewendeten, \gls{nKE} (alle fünf Einheitsschritte) aufgenommen. 

Auf der Abszisse des erhaltenen Diagramms befindet sich die \gls{nKE} und auf der Ordinate die Intensität der jeweiligen Fragmentierung bezogen auf den höchsten Peak, der während der Aufnahme beobachtet wurde (ebenfalls in Prozent). 

Die erhaltenen Kurven wurden mit einem Savitzky-Golay Filter geglättet (siehe Anhang) und werden im folgenden als Fragmentierungsdiagramme bezeichnet. Ein Nachteil bei der Behandlung mit diesem Filter ist, dass in manchen Fällen die Intensitäten der Fragmentierungen bei einer normalisierten Kollisionsenergie von null nicht gleich null sind. Es wird im folgenden angenommen, dass dies dennoch so ist.

Aufgenommen wurden die Fragmentierungsdiagramme im \gls{cid} und im \gls{pqd} Modus.\\

Es wird damit versucht, herauszufinden, ob bestimmte Abspaltungen der Kataboliten charakteristische Muster aufweisen, um in weiterer Hinsicht weitere strukturelle Eigenschaften über die Kataboliten mithilfe eines Massenspektrometers zu erfahren. Weiters wird ein Vergleich zwischen den Fragmentierungsdiagrammen von MS Leafspray und des hochauflösenden Massenspektrometers versucht. Für diesen Vergleich wurden nur die Diagramme verwendet, die im \gls{cid} Modus aufgenommen wurden, da das verwendete Massenspektrometer von MS-Leafspray nur in diesem Modus operieren konnte.