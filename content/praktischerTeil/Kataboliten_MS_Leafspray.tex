\section{Chl-Kataboliten des Brokkoliblattes mithilfe von MS-Leafspray identifiziert}

Im Folgenden werden die \gls{Chl-K}en beschrieben, die sich durch MS Leafspray identifizieren ließen. Die Strukturvorschläge wurden mit einem hochauflösendem Massenspektrometer überprüft (Kapitel \ref{sec:ChlKatabolitenBrokkoli} und \ref{sec:ChlKatabolitenESIMS}). Sie beruhen auf den exakten Molekülmassen und den daraus errechneten möglichen Summenformeln. Eine exakte Strukturaufklärung müsste mit \textsuperscript{1}H-NMR durchgeführt werden. 

Fragmentierungsdiagramme wurden wie in Kapitel \ref{sec:fragmentierungsdiagramme} beschrieben erstellt.

\subsection{Bo-NCC-1} \label{sec:MSLeafsprayBoNCC1}

Bei diesem Kataboliten handelt es sich vermutlich um denselben, wie er auch in der Brokkolifrucht gefunden wurde, weswegen er die Bezeichnugn Bo-NCC-1 erhält. \cite{ChlorophyllCatabolitesBroccoli} Beobachtet wurde die protonierte Verbindung bei m/z = 793 [M+H]\textsuperscript{+} und das Kaliumsalz bei m/z = 831 [M+K]\textsuperscript{+} (Abbildung \ref{fig:831MKLeafspray}). Aufgrund der geringen Intensitäten der protonierten Verbindung war es nicht möglich, ein verwertbares Massenspektrum dieser aufzunehmen. \\

Der Katabolit bei m/z = 831 [M+K]\textsuperscript{+} zeigte Abspaltungen von \ch{H2O} bei m/z = 813 [M - \ch{H2O} + K]\textsuperscript{+}, von \ch{CO2} bei m/z = 787 [M - \ch{CO2} + K]\textsuperscript{+} und eine Folge von Abspaltungen bei m/z = 311 [M - (Ring A, Zucker, Ring D, \ch{CO2}) + K]\textsuperscript{+}, bei der Ring A mit einem Zucker, Ring D sowie \ch{CO2} abgespalten wird (siehe Kapitel \ref{sec:ChlKatabolitenESIMS}). Die Abspaltungen bei m/z = 798 [M - (\gls{nAb}) + K]\textsuperscript{+}, m/z = 586 [M - (\gls{nAb}) + K]\textsuperscript{+} und m/z = 551 [M - (\gls{nAb}) + K]\textsuperscript{+} können nicht eindeutig zugeordnet werden, da hierzu weitere experimentelle Daten und vor allem die exakten Molekülmassen vonnöten sind. 

\begin{figure}[!htbp]
  \includegraphics[width=\textwidth, height=0.7\textwidth]{figures/Kapitel4/Kataboliten/VWA_MS_LeafSpray_831.png}
  \caption[ESI-MS Spektrum von Bo-NCC-1, Quelle: Autor]{ESI-MS von Bo-NCC-1 mit m/z = 831 [M+K]\textsuperscript{+}}
  \label{fig:831MKLeafspray}
\end{figure}

Das Fragment bei m/z = 798 [M - (\gls{meoh}?) + K]\textsuperscript{+} ist insofern interessant, da es sich hierbei um eine Abspaltung von \gls{meoh} (-32 Da) handeln könnte (es wird angenommen, dass die Abweichung um eine Einheit durch Ungenauigkeiten des Massenspektrometers zustandekommt), was aber nicht mit der Struktur des Bo-NCC-1 (siehe Abbildung \ref{fig:831MKLeafspraystructure}) vereinbar wäre. Aufgrund ihrer Fragwürdigkeit wird auf diese Abspaltung in den weiteren Ausführungen nicht näher eingegangen.

Wie aus dem Fragmentierungsdiagramm (Abbildung \ref{fig:831MKLeafspraydiags}) ersichtlich, erfolgt die Abspaltung von \ch{H2O} bei einer niedrigeren \gls{nKE} wie jene von \ch{CO2} und verschwindet bei höheren Energien, wohingegen die Abspaltung von \ch{CO2} erhalten bleibt. Die Abspaltung von \ch{H2O} erreicht ein lokales Maximum bei einer \gls{nKE} von 10. Die Abspaltung von \ch{CO2} erreicht ein lokales Maximum bei 30 \gls{nKE}.

Aufgrund der \ch{CO2} Abspaltung wird an Position 8\textsuperscript{2} eine Carbonsäuregruppe vermutet (wie in \cite{StructureElucidation} gezeigt), die über einen Mechanismus wie unter anderem (u. a.) in Abbildung \ref{fig:619MHElectronMovement} vorgeschlagen, abgespalten wird. Die relativ große Molekülmasse weist zudem auf einen Zucker an Position 32 hin. Die Summenformel des Bo-NCC-1 konnte über die exakte Molekülmasse mit einem hochauflösenden Massenspektrometer bestimmt werden (Kapitel \ref{sec:ESIMSBoNCC1}).

\begin{figure}[!htbp]
  \begin{subfigure}[b]{0.5\textwidth}
    \includegraphics[width=\textwidth]{figures/Kapitel4/Kataboliten/fragmentation_structures/VWA_Katabolit_831.png}
    \caption{}
    \label{fig:831MKLeafspraystructure}
  \end{subfigure}
  \hfill
  \begin{subfigure}[b]{0.7\textwidth}
    \includegraphics[width=\textwidth]{figures/Kapitel4/Kataboliten/diags/831CID-savgol.png}
    \caption{}
    \label{fig:831MKLeafspraydiags}
  \end{subfigure}
  \caption[Strukturvorschlag von Bo-NCC-1 und Fragmentierungsdiagramm, Quelle: Autor]{(a) Strukturvorschlag des Bo-NCC-1 mit Summenformel \ch{C40H48N4O13}, (b) Fragmentierungsdiagramm von Bo-NCC-1 (blau = 831 [M+K]\textsuperscript{+}, orange = 813 [M - \ch{H2O} + K]\textsuperscript{+}, grün = 798 [M - (\ch{MeOH} - \gls{nAb}) + K]\textsuperscript{+}, rot = 787 [M - \ch{CO2} + K]\textsuperscript{+})}
\end{figure}



\subsection{Bo-NCC-3}

Beim Bo-NCC-3 handelt es sich um einen \gls{Chl-K}, der bisher nicht in der Brokkolifrucht identifiziert wurde \cite{ChlorophyllCatabolitesBroccoli}, weswegen er als dritter, in der Brokkolipflanze gefundener Katabolit den Index 3 erhält. Analysiert wurde das Kaliumsalz mit m/z = 685 [M+K]\textsuperscript{+}. \\

Es wurden zwei charakteristische Abspaltungen von \ch{H2O} bei m/z = 667 [M - \ch{H2O} + K]\textsuperscript{+} sowie von \ch{CO2} bei m/z = 641 [M - \ch{CO2} + K]\textsuperscript{+} beobachtet. Bei den Abspaltungen bei m/z = 429 [M - (\gls{nAb}) + K]\textsuperscript{+}, m/z = 561 [M - (\gls{nAb}) + K]\textsuperscript{+}, m/z = 605 [M - (\gls{nAb}) + K]\textsuperscript{+} und m/z = 652 [M - (\gls{nAb}) + K]\textsuperscript{+} ist nicht eindeutig geklärt, welche Fragmente hierbei entstanden sind. Für das Fragment bei m/z = 652 [M - (\gls{meoh}?) + K]\textsuperscript{+} gilt dasselbe wie bei der Abspaltung von m/z = 798 [M - (\ch{MeOH}?) + K]\textsuperscript{+} von Bo-NCC-1 (Kapitel \ref{sec:MSLeafsprayBoNCC1}). Um diese Fragmente aufzuklären müssten weitere Experimente des Kaliumsalzes mit einem hochauflösenden Massenspektrometer durchgeführt werden. Fragmentierungen der protonierten Verbindung konnten mit einem hochauflösenden Massenspektrometer gemessen werden (Kapitel \ref{sec:ESIMSBoNCC3}).

\begin{figure}[htbp]
  \includegraphics[width=\textwidth, height=0.7\textwidth]{figures/Kapitel4/Kataboliten/VWA_MS_LeafSpray_685.png}
  \label{fig:685MKLeafspray}
  
  \caption[ESI-MS von Bo-NCC-3, Quelle: Autor]{ESI-MS von Bo-NCC-3 mit m/z = 685 [M+K]\textsuperscript{+}}
\end{figure}

\begin{figure}[!htbp]
  \begin{subfigure}[b]{0.4\textwidth}
    \includegraphics[width=\textwidth]{figures/Kapitel4/Kataboliten/fragmentation_structures/VWA_Katabolit_685.png}
    \caption{}
    \label{fig:685MKLeafspraystructure}
  \end{subfigure}
  \hfill
  \begin{subfigure}[b]{0.7\textwidth}
    \includegraphics[width=\textwidth]{figures/Kapitel4/Kataboliten/diags/685CID-savgol.png}
    \caption{}
    \label{fig:685MKLeafspraydiags}
  \end{subfigure}
  \caption[Strukturvorschlag von Bo-NCC-3 und Fragmentierungsdiagramm, Quelle: Autor]{(a) Strukturvorschlag von Bo-NCC-3 mit Summenformel \ch{C34H38N4O9}, (b) Fragmentierungsdiagramm von Bo-NCC-3 (blau = 685 [M+K]\textsuperscript{+}, orange = 667 [M - \ch{H2O} + K]\textsuperscript{+}, grün = 652 [M - (\gls{meoh}?) + K]\textsuperscript{+}, rot = 641 [M - \ch{CO2} + K]\textsuperscript{+}, violett = 605 [M - (\gls{nAb}) + K]\textsuperscript{+})}
\end{figure}

Das Fragmentierungsdiagramm zeigt, dass die Abspaltung von \ch{H2O} bei einer niedrigeren \gls{nKE} erfolgt, wie jene von \ch{CO2}, da sie ihre höchste Intensität zuvor erreicht (bei einer \gls{nKE} von 15 - \ch{H2O} im Vergleich zu 20 - \ch{CO2}). 

Im Vergleich zum Bo-NCC-1 zeigt der Graph ein lokales Maximum der \ch{H2O} Abspaltung bei höheren Energien (beim Bo-NCC-3 bei 15 \gls{nKE} wohingegen beim Bo-NCC-1 bereits bei 10 \gls{nKE}). Das lokale Maximum der \ch{CO2} Abspaltung verschiebt sich von 30 \gls{nKE} beim Bo-NCC-1 auf 20 \gls{nKE} beim Bo-NCC-3. Das lokale Maximum der potentiellen Abspaltung von \gls{meoh} würde sich von 25 \gls{nKE} beim Bo-NCC-1 auf 30 \gls{nKE} beim Bo-NCC-3 verschieben (Abbildungen \ref{fig:831MKLeafspraydiags} und \ref{fig:685MKLeafspraydiags}).\\ 

Wie beim Bo-NCC-1 weist die \ch{CO2} Abspaltung auf eine freie Carbonsäure an Position 8\textsuperscript{2} hin. Aufgrund der durch die Summenformel erhaltene Sauerstoffanzahl wird angenommen, dass sich an Position 15 eine Hydroxygruppe befindet (Abbildung \ref{fig:685MKLeafspraystructure}). Es wird vermutet, dass es sich dabei um eine Vorstuffe zu einem \gls{YCC} handelt. [Referenz]

\subsection{Bo-DNCC}

Es wird vermutet, dass der Bo-DNCC des Brokkoliblattes ident ist mit dem Bo-DNCC der Brokkolifrucht. \cite{ChlorophyllCatabolitesBroccoli} Beobachtet wurden zwei Pseudo-Molekulare Ionen. Eines mit m/z = 619 [M+H]\textsuperscript{+} (Abbildung \ref{fig:619MHLeafspray}) und mit m/z = 657 [M+K]\textsuperscript{+} (Abbildung \ref{fig:657MKLeafspray}).\\

Der Katabolit bei m/z = 619 [M+H]\textsuperscript{+} zeigte Abspaltungen von \ch{H2O} bei m/z = 601 [M - \ch{H2O} + H]\textsuperscript{+}, von \ch{CO2} bei m/z = 575 [M - \ch{H2O} + H]\textsuperscript{+}, von Ring D (zusammen mit einer Abspaltung von \ch{CO2}) bei m/z = 452 [M - (Ring D, \ch{CO2}) + H]\textsuperscript{+} und von Ring A, Ring D und \ch{CO2} bei m/z = 311 [M - (Ring A, Ring D, \ch{CO2}) + H]\textsuperscript{+} (Abbildung \ref{fig:619MHLeafspray} - Zuordnung Kapitel \ref{sec:ESIMSBoDNCC}).

\begin{figure}[!htbp]
  \includegraphics[width=\textwidth, height=0.7\textwidth]{figures/Kapitel4/Kataboliten/VWA_MS_LeafSpray_619.png}
  \caption[ESI-MS von Bo-DNCC, Quelle: Autor]{ESI-MS von Bo-DNCC bei m/z = 619 [M+H]\textsuperscript{+}}
  \label{fig:619MHLeafspray}
\end{figure}

Das Kaliumsalz des Bo-DNCC mit m/z = 657 [M+K]\textsuperscript{+} zeigte eindeutige Abspaltungen von \ch{H2O} bei m/z = 639 [M - \ch{H2O} + K]\textsuperscript{+} und von \ch{CO2} bei m/z = 613 [M - \ch{CO2} + K]\textsuperscript{+} (Abbildung \ref{fig:657MKLeafspray}). Die Abspaltungen bei m/z = 375 [M - (\gls{nAb}) + K]\textsuperscript{+} und m/z = 577 [M - (\gls{nAb}) + K]\textsuperscript{+} können nicht eindeutig zugeordnet werden.

\begin{figure}[!htbp]
  \includegraphics[width=\textwidth, height=0.7\textwidth]{figures/Kapitel4/Kataboliten/VWA_MS_LeafSpray_657.png}
  \caption[ESI-MS von Bo-DNCC, Quelle: Autor]{ESI-MS von Bo-DNCC bei m/z = 657 [M+K]\textsuperscript{+}}
  \label{fig:657MKLeafspray}
\end{figure}

\begin{figure}[!htbp]
  \begin{subfigure}[b]{0.4\textwidth}
    \includegraphics[width=\textwidth]{figures/Kapitel4/Kataboliten/fragmentation_structures/VWA_Katabolit_619.png}
    \caption{}
    \label{fig:619MKLeafspraystructure}
  \end{subfigure}
  \hfill
  \begin{subfigure}[b]{0.7\textwidth}
    \includegraphics[width=\textwidth]{figures/Kapitel4/Kataboliten/diags/619CID-savgol.png}
    \caption{}
    \label{fig:619MKLeafspraydiags}
  \end{subfigure}
  \caption[Strukturvorschlag von Bo-DNCC mit Fragmentierungsdiagramm, Quelle: Autor]{(a) Strukturvorschlag von Bo-DNCC mit Summenformel \ch{C33H38N4O8}, (b) Fragmentierungsdiagramm von Bo-DNCC (blau = 619 [M+H]\textsuperscript{+}, orange = 601 [M - \ch{H2O} + H]\textsuperscript{+}, grün = 575 [M - \ch{CO2} + H]\textsuperscript{+})}
\end{figure}

Die \ch{H2O} Abspaltung beim Bo-DNCC erreicht ein lokales Maximum bei 20 \gls{nKE} und erfolgt damit im Vergleich zum Bo-NCC-1 und Bo-NCC-3 bei der höchsten \gls{nKE}. Die Abspaltung von \ch{CO2} weist beim Bo-DNCC zwei lokale Maxima, bei 25 \gls{nKE} und 75 \gls{nKE} auf. Das lokale Maximum an der Stelle 75 \gls{nKE} ist dabei etwas weniger intensiv ausgeprägt wie jenes an der Stelle 25 \gls{nKE}. Das erste lokale Maximum befindet sich damit an der gleichen Stelle wie bei Bo-NCC-1 und Bo-NCC-3 (Abbildungen \ref{fig:831MKLeafspraydiags} und \ref{fig:685MKLeafspraydiags}). Das zweite Maximum kann noch nicht geklärt werden, da es bei den anderen bisher analysierten Kataboliten nicht beobachtet wurde.

Es ist fraglich, ob der Vergleich mit den Fragmentierungsdiagrammen von Bo-NCC-1 und Bo-NCC-3 möglich ist, da bei diesen das [M+K]\textsuperscript{+} Ion aufgenommen wurde.
