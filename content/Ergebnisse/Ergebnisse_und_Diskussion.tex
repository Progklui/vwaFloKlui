\chapter{Ergebnisse}


\section{Liste aller identifizierten Chl-Kataboliten und deren Reaktionsprodukten}

In Tabelle \ref{tab:LCMSKatabolitenRPListe} werden alle Verbindungen aufgelistet, die ich im Rahmen meiner Arbeit in den diversen Experimenten identifizieren konnte. Es folgt ein Index zur Tabelle:

\begin{description}[align=right,labelwidth=3cm]
  \item [Chl-Katabolit] . . . vorgeschlagene Bezeichnung für mit großer Sicherheit identifizierte Chl-Kataboliten
  \item [Summenformel] . . . die Summenformel des [M+H]\textsuperscript{+} Ions, gemessen in ESI-MS und LC-MS Experimenten - bei mit * gekennzeichneten handelt es sich um die Summenformel des ungeladenen Chl-Kataboliten.
  \item [M+H] . . . Molekülmasse des [M+H]\textsuperscript{+} Ions, die in ESI-MS oder LC-MS Experimente gemessen wurde
  \item [Frag. I] . . . gibt an, ob bei den ESI-MS Experimenten ein Fragmentierungsdiagramm erstellt wurde
  \item [MS-Leafspray] . . . Molekülmasse des [M+K]\textsuperscript{+} Ions, sofern mit MS-Leafspray gemessen
  \item [Frag. II] . . . gibt an, ob bei den MS-Leafspray Experimenten ein Fragmentierungsdiagramm erstellt wurde
  \item [Typ] . . . gibt die Art des Chl-Kataboliten an, sofern dies klar ist - bei mit * gekennzeichneten handelt es sich um Anhydride
  \item [HPLC] . . . gibt an, ob Chl-Katabolit mithilfe von UV/Vis Spektren in der HPLC identifiziert werden konnte
  \item [H.] . . . (Herkunft) gibt an, ob es sich bei der Verbindung um ein Reaktionsprodukt handelt und wenn ja, welcher Chl-Katabolit das Edukt war
\end{description}

In dieser Tabelle werden somit die Ergebnisse in übersichtlicher Form festgehalten.

\begin{sidewaystable*}[!htbp]\centering
  \ra{1.3}
  
  \begin{tabular}{ccccccccc}\toprule
 Chl-Katabolit & Summenformel & [M+H]\textsuperscript{+} & Frag. I & MS-Leafspray & Frag. II & Typ & HPLC & H. \\
\midrule
\rowcolor{black!20} Bo-DYCC & \ch{C33H37O8N4} & 617.2599 & \checkmark & x & x & DYCC & 30.94? & -\\
 Bo-DNCC & \ch{C33H39O8N4} & 619.2798 & \checkmark & 657 & \checkmark & DNCC & 26.72 & -\\ 
\rowcolor{black!20} • & \ch{C34H37O8N4} & 629.2641 & x & x & x & • & - & -\\ 
 - & \ch{C34H39O8N4} & 631.2795 & \checkmark & x & x & DYCC & 29.91, 30.94 & Bo-DYCC\\ 
\rowcolor{black!20} - & \ch{C34H41O8N4} & 633.2955 & \checkmark & x & x & DNCC & 28.8 & Bo-DNCC\\ 
 • & \ch{C36H33O7N4} & 633.2339 & x & x & x & • & - & -\\ 
\rowcolor{black!20} Bo-YCC & \ch{C34H37O9N4} & 645.2593 & \checkmark & x & x & YCC & - & -\\ 
 - & \ch{C35H41O8N4} & 645.2953 & x & x & x & DYCC & - & Bo-DYCC\\ 
\rowcolor{black!20} Bo-NCC-3 & \ch{C34H39O9N4} & 647.2748 & \checkmark & 685 & \checkmark & NCC & 33.04 & -\\ 
 • & \ch{C34H35O10N4} & 659.2348 & x & x & x & • & - & -\\
\rowcolor{black!20} - & \ch{C35H39O9N4} & 659.2741 & \checkmark & x & x & YCC & 37.09 & Bo-YCC\\
 - & \ch{C35H41O9N4} & 661.2902 & \checkmark & x & x & NCC & - & Bo-NCC-3\\
\rowcolor{black!20} - & \ch{C36H43O9N4} & 675.306 & \checkmark & x & x & NCC & - & Bo-NCC-3\\
 Bo-DNCC-2 & \ch{C39H47O13N4} & 779.3181 & x & x & x & DNCC & - & -\\ 
\rowcolor{black!20} Bo-NCC-1 & \ch{C40H49O13N4} & 793.3336 & \checkmark & 831 & \checkmark & NCC & 29.91 & -\\ 
 - & \ch{C40H51O13N4} & 795.3491 & \checkmark & x & x & - & - & -\\ 
\rowcolor{black!20} - & \ch{C41H51O13N4} & 807.3491 & \checkmark & x & x & NCC & 40.03 & Bo-NCC-1\\ 
 - & \ch{C41H53O13N4} & 809.3649 & x & x & x & - & - & 795\\ 
\rowcolor{black!20} - & \ch{C42H53O13N4} & 821.3652 & x & x & x & NCC & 47.28 & Bo-NCC-1\\ 
 - & \ch{C33H40N4O9}* & x & x & 699 & \checkmark & DNCC* & - & Bo-DNCC \\ 
\rowcolor{black!20} - & \ch{C36H40N4O1}* & x & x & 727 & \checkmark & NCC* & & Bo-NCC-3\\ 
 - & \ch{C42H50N4O14}* & x & x & 873 & \checkmark & NCC* & - & Bo-NCC-1 \\ 
\bottomrule
  \end{tabular}
  
  \caption[Übersicht über die Chl-Kataboliten des Brokkoliblattes unter Verwendung aller Methoden, Quelle: Autor]{Übersicht über die gefundenen Chl-Kataboliten des Brokkoliblattes und ihren Reaktionsprodukten}
  \label{tab:LCMSKatabolitenRPListe}
\end{sidewaystable*}
