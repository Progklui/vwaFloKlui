\chapter{Ergebnisse}

Da die Ergebnisse bereits im experimentellen Teil ausführlich behandelt worden sind, folgt hier nur eine Übersicht über die identifizierten Chl-Kataboliten.

\section{Auflistung aller identifizierten Chl-Kataboliten und deren Reaktionsprodukte}

In Tabelle \ref{tab:LCMSKatabolitenRPListe} werden alle Verbindungen aufgelistet, die ich im Rahmen meiner Arbeit in den diversen Experimenten identifizieren konnte. Index zur Tabelle:

\begin{description}[align=right,labelwidth=3cm]
  \item [Chl-Katabolit] . . . vorgeschlagene Bezeichnung für mit großer Sicherheit identifizierte Chl-Kataboliten
  \item [Summenformel] . . . die Summenformel des [M+H]\textsuperscript{+} Ions, gemessen in ESI-MS und LC-MS Experimenten - bei mit * gekennzeichneten handelt es sich um die Summenformel des ungeladenen Chl-Kataboliten.
  \item [M+H] . . . Molekülmasse des [M+H]\textsuperscript{+} Ions, die in ESI-MS oder LC-MS Experimente gemessen wurde
  \item [Frag. I] . . . gibt an, ob bei den ESI-MS Experimenten ein Fragmentierungsdiagramm erstellt wurde
  \item [MS-Leafspray] . . . Molekülmasse des [M+K]\textsuperscript{+} Ions, sofern mit MS-Leafspray gemessen
  \item [Frag. II] . . . gibt an, ob bei den MS-Leafspray Experimenten ein Fragmentierungsdiagramm erstellt wurde
  \item [Typ] . . . gibt die Art des Chl-Kataboliten an, sofern dies klar ist - bei mit * gekennzeichneten handelt es sich um Anhydride
  \item [HPLC] . . . gibt an, ob Chl-Katabolit mithilfe von UV/Vis Spektren in der HPLC identifiziert werden konnte
  \item [H.] . . . (Herkunft) gibt an, ob es sich bei der Verbindung um ein Reaktionsprodukt handelt und wenn ja, welcher Chl-Katabolit das Edukt war
\end{description} 


\begin{sidewaystable*}[!htbp]\centering
  \ra{1.3}
  
  \begin{tabular}{ccccccccc}\toprule
 Chl-Katabolit & Summenformel & [M+H]\textsuperscript{+} & Frag. I & MS-Leafspray & Frag. II & Typ & HPLC & H. \\
\midrule
\rowcolor{black!20} Bo-DYCC & \ch{C33H37O8N4} & 617.2599 & \checkmark & x & x & DYCC & 30.94? & -\\
 Bo-DNCC & \ch{C33H39O8N4} & 619.2798 & \checkmark & 657 & \checkmark & DNCC & 26.72 & -\\ 
\rowcolor{black!20} • & \ch{C34H37O8N4} & 629.2641 & x & x & x & • & - & -\\ 
 - & \ch{C34H39O8N4} & 631.2795 & \checkmark & x & x & DYCC & 29.91, 30.94 & Bo-DYCC\\ 
\rowcolor{black!20} - & \ch{C34H41O8N4} & 633.2955 & \checkmark & x & x & DNCC & 28.8 & Bo-DNCC\\ 
 • & \ch{C36H33O7N4} & 633.2339 & x & x & x & • & - & -\\ 
\rowcolor{black!20} Bo-YCC & \ch{C34H37O9N4} & 645.2593 & \checkmark & x & x & YCC & - & -\\ 
 - & \ch{C35H41O8N4} & 645.2953 & x & x & x & DYCC & - & Bo-DYCC\\ 
\rowcolor{black!20} Bo-NCC-3 & \ch{C34H39O9N4} & 647.2748 & \checkmark & 685 & \checkmark & NCC & 33.04 & -\\ 
 • & \ch{C34H35O10N4} & 659.2348 & x & x & x & • & - & -\\
\rowcolor{black!20} - & \ch{C35H39O9N4} & 659.2741 & \checkmark & x & x & YCC & 37.09 & Bo-YCC\\
 - & \ch{C35H41O9N4} & 661.2902 & \checkmark & x & x & NCC & - & Bo-NCC-3\\
\rowcolor{black!20} - & \ch{C36H43O9N4} & 675.306 & \checkmark & x & x & NCC & - & Bo-NCC-3\\
 Bo-DNCC-2 & \ch{C39H47O13N4} & 779.3181 & x & x & x & DNCC & - & -\\ 
\rowcolor{black!20} Bo-NCC-1 & \ch{C40H49O13N4} & 793.3336 & \checkmark & 831 & \checkmark & NCC & 29.91 & -\\ 
 - & \ch{C40H51O13N4} & 795.3491 & \checkmark & x & x & - & - & -\\ 
\rowcolor{black!20} - & \ch{C41H51O13N4} & 807.3491 & \checkmark & x & x & NCC & 40.03 & Bo-NCC-1\\ 
 - & \ch{C41H53O13N4} & 809.3649 & x & x & x & - & - & 795\\ 
\rowcolor{black!20} - & \ch{C42H53O13N4} & 821.3652 & x & x & x & NCC & 47.28 & Bo-NCC-1\\ 
 - & \ch{C35H41N4O9}* & x & x & 699 & \checkmark & DNCC* & - & Bo-DNCC \\ 
\rowcolor{black!20} - & \ch{C36H40N4O10}* & x & x & 727 & \checkmark & NCC* & & Bo-NCC-3\\ 
 - & \ch{C42H50N4O14}* & x & x & 873 & \checkmark & NCC* & - & Bo-NCC-1 \\ 
\bottomrule
  \end{tabular}
  
  \caption[Übersicht über die Chl-Kataboliten des Brokkoliblattes unter Berücksichtigun der Erkenntnisse aller Methoden, Quelle: Autor]{Übersicht über die gefundenen Chl-Kataboliten des Brokkoliblattes und ihren Reaktionsprodukten}
  \label{tab:LCMSKatabolitenRPListe}
\end{sidewaystable*}

\chapter{Diskussion}

Im Folgenden werden die im Experimentellen Teil beschriebenen Ergebnisse in Hinblick auf die in Kapitel \ref{sec:Themenstellung} gestellten Fragen analysiert. Es werden dabei nur die wichtigsten, sogenannte Highlights herausgenommen, da eine genauere Ausführung den Rahmen sprengen würde.

\section{zur direkten Analyse, MS Leafspray und der Reaktion mit Essigsäureanhydrid}

Wie in Kapitel \ref{sec:MSLeafspray} ersichtlich, konnten die \gls{Chl-K}en erfolgreich in einer direkten Analyse mithilfe von MS Leafspray identifiziert werden.  

Zu Beginn waren die Experimente jedoch von deutlichen Misserfolgen geprägt. Es scheiterte z.B. bereits daran, ein schönes, konstantes Signal im Massenspektrometer zu bekommen. Auch ein Optimieren des Signals durch Kalibrierung und andere Methoden brachte keine Besserung. Erst die Entwicklung der in Kapitel \ref{sec:MSLeafspray} beschriebenen Blattvorbereitungstechnik(en) konnte zusammen mit der Erkenntnis, dass man gelegentlich vom positiven in den negativen Ionenmodus schalten muss zu schönen und vor allem verwertbaren Massenspektren und Fragmentierungsdiagrammen führen\footnote{diese Erkenntnis ist einem Zufall zu verdanken: während dem Messen verschwand das Signal plötzlich und ich befürchtete schon, das Massenspektrometer verstopft zu haben (eine Katastrophe) - nach einem mehr oder weniger zufälligem Umschalten in den negativen Ionenmodus und dem sofortigen Zurückschalten in den positiven war das Signal wieder in vollster Intensität vorhanden}. 

Nachdem die \gls{Chl-K}en des Brokkoliblattes erfolgreich identifiziert wurden, konnten mit der gleichen Methode auch die Reaktionsprodukte der \gls{Chl-K}en mit Essigsäureanhydrid gemessen werden. Unter Verwendung von Essigsäureanhydrid als LM konnten dabei sogar die Anhydride als Reaktionsprodukte identifiziert werden.

Zur Reaktivität der Carbonsäuren wurde beobachtet, dass die freie Carbonsäure an Position 8\textsuperscript{2} deutlich reaktiver ist wie jene an Position 12\textsuperscript{2}. Vermutlich ist dies durch die bevorzugte sterische Umgebung begründet, wobei dies bisher reine Spekulation ist.

\section{zu den Fragmentierungsdiagrammen und Strukturaufklärung}

Es konnten für alle mit MS Leafspray gemessenen Verbindungen Fragmentierungsdiagramme erfolgreich aufgenommen werden. In der Arbeit wurde vorgeschlagen, dass eine Analyse dieser über den Vergleich von Minima und Maxima möglich ist. Auf diese Weise kann ein Einblick auf massenspektrometrische Eigenschaften der Verbindungen gewonnen werden. 

Es wird vermutet, dass die Fragmentierungsdiagramme charakteristisch für einen Chl-Kataboliten sind, weil diesselben Diagramme bei zu unterschiedlichen Zeitpunkten durchgeführten Experimenten erstellt wurden. Es handelt sich dabei jedoch nur um eine Theorie, die noch in weiterer, gezielter experimenteller Arbeit verifiziert werden müsste.

Um einen Blick in die Zukunft zu wagen, könnte man diese Fragmentierungsdiagramme verwenden, um eine Schnellidentifikation von Chl-Kataboliten durchzuführen. Ein Anwendungsgebiet wäre hier beispielsweise die Landwirtschaft, in der ein Bauer mithilfe von MS Leafspray \textit{am Feld} Chl-Kataboliten identifizieren könnte. Um diese Schnellidentifikation zu programmieren, könnte man auf die Methode des Deep-Learnings zurückgreifen, die ein Analysieren der Diagramme erleichtern und vor allem automatisieren könnte. 

Unter Verwendung eines solchen Deep-Learning Netwerkes ist es außerdem denkbar, dass man aus den Fragmentierungsdiagramm weitere Strukturmerkmale \textit{herauskitzeln} kann, was einen Fortschritt in Bezug auf Strukturaufklärung mit dem Massenspektrometer bedeuten würde.

Außerdem ist denkbar, dass über eine intensivere Erforschung der Fragmentierungsdiagramme ein tieferer Einblick in die Eigenschaften der chemischen Bindung in komplexen Molekülen gewonnen werden kann. Auch hier ist eine Anwendung von DeepLearning denkbar.\\

Im Folgenden wird eine Theorie zur Interpretation der im CID-Modus aufgenommenen Fragmentierungsdiagramme entwickelt. Im Massenspektrometer kommt es durch teilweise Umwandlung kinetischer Energie in intramolekulare Energie zur  Anregung des Molekül und dementsprechend zu Schwingungen desselben. Diese Schwingungen führen letztendlich zum Bindungsbruch und zu den messbaren Fragmenten. Bei der Erstellung der Fragmentierungsdiagramme wurde die Intensität dieser Fragmente zur aufgewendeten \gls{nKE} gemessen. \\

Was bei der Betrachtung der Diagramme auffällt, ist, dass sie Minima und Maxima besitzen, die im Praktischen Teil der Arbeit ausführlich beschrieben worden sind. Außerdem nehmen die Intensitäten bei höheren Energien zumeist ab. Die meisten Maxima befinden sich in einem Bereich von 10 bis 50 \gls{nKE}. 

\gls{Chl-K}en sind von ihrer räumlichen Struktur komplexe Moleküle und können aufgrund von Drehungen um Einfachbindungen diverse räumliche Strukturen annehmen, die aufgrund von sterischen Effekten thermodynamisch nicht gleich stabil sind. Wenn ein solcher \gls{Chl-K} energetisch angeregt wird und demnach schwingt, gibt es vermutlich bestimmte energetisch stabile räumliche Anordnungen, die abhängig von der \gls{nKE} sind. Das könnte bedeuten, dass es bestimmte \gls{nKE}s gibt, die den \gls{Chl-K}en in bestimmte räumliche Anordnungen bringen, bei denen eine Abspaltung eines Ringes beispielsweise energetisch günstig ist. 

Nach diesem Modell wäre bei einer großen \gls{nKE} die Anregung so groß, dass es zu sterischen intramolekularen Hinderungen kommt, die Abspaltungen entweder gänzlich verhindern oder das Molekül zum unkontrollierten Auseinanderbrechen bringt. Das Entstehen von Maxima einer Abspaltung kann durch das Einnehmen eines für diese Abspaltung energetisch günstigen Zustandes erklärt werden. Minima könnten das Einnehmen eines Übergangszustandes beschreiben.

Bei der vorliegenden Betrachtung werden die Intensitäten der Abspaltungen aufgrund der unterschiedlichen Ionisierbarkeit nicht berücksichtigt. Es wird angenommen, dass die Ionisierbarkeit nicht von der \gls{nKE} abhängt.


\section{zum Vergleich direkte Analyse - klassischer Ansatz}

In Tabelle \ref{tab:ComparisonDirectClassic} werden die wesentlichen Unterschiede der beiden Analysemethoden, die ich während meiner Arbeit beobachten konnte, herausgearbeitet. Die Liste stellt nicht den Anspruch, vollständig zu sein.

\newcolumntype{C}[1]{>{\centering\arraybackslash}p{#1}}

\begin{table*}[!htbp]\centering
  \ra{1.3}
  
  \begin{tabular}{C{1cm}C{6cm}C{1cm}C{6cm}}\toprule
 +/- & direkte Analyse & +/- & klassischer Ansatz \\
\midrule
\rowcolor{black!20} + & kurze Blattvorbereitungszeit für die Analyse & - & lange Blattvorbereitungszeit für die Analyse (Aufreibung, Abzentrifugieren, ...) \\
 + & lange Analysezeiten, die z.B. für die Erstellung von Fragmentierungsdiagrammen verwendet werden kann & - & 75 min. Zeit vergeht, bis analysiert werden kann - erzeugen von Fragmentierungsdiagrammen nur über Sammeln in EPPIs möglich \\ 
\rowcolor{black!20} + & erlaubt einfache Isolierung von Anhydriden durch Verwendung von Acetonitril als LM & - & Isolierung von Anhydriden als Reaktionsprodukt schwer möglich, da bevorzugtes LM MeOH \\
 + & erlaubt Analyse von vielen Blättern in kurzer Zeit (ca. 30 min. pro Blatt, wenn man viele Chl-Kataboliten analysiert) & - & (fast) vollständige Analyse eines Blattes dauert bis zu 100 min., sofern sehr effizient gearbeitet wird \\ 
\rowcolor{black!20} - & man bekommt kein Chromatogramm, über das man alle Chl-Kataboliten potentiell herausfinden kann - es besteht die Gefahr, dass einige Chl-Kataboliten nicht analysiert werden & + & höhere Wahrscheinlichkeit des Auffindens von Chl-Kataboliten \\
 - & bevorzugt [M+K]\textsuperscript{+} Ionen & - & bevorzugt [M+H]\textsuperscript{+} Ionen \\ 
\rowcolor{black!20} - & Molekülmasse nur auf 1 signifikante Kommastelle messbar - Summenformel kann nicht berechnet werden & + & Molekülmasse auf 3 signifikante Kommastellen messbar - Möglichkeit, Summenformel zu berechnen \\
 + & theoretisch relativ einfach automatisierbar & - & aufwändig zu automatisieren \\ 
 \rowcolor{black!20} - & Massenspektrometer kann mit der Zeit verschmutzen & + & Massenspektrometer nicht so anfällig für ein Verschmutzen \\
\bottomrule
  \end{tabular}
  
  \caption[Vergleich beider Methoden, Quelle: Autor]{Vergleich von direkter Analyse mit einer klassischen Analyse}
  \label{tab:ComparisonDirectClassic}
\end{table*}

\pagebreak
\section{zum Vergleich Brokkoliblatt - Brokkolifrucht}

In meiner Arbeit wurden die Chl-Kataboliten des Brokkoliblattes untersucht. Im Vergleich zu einer kürzlich veröffentlichten Publikation \cite{ChlorophyllCatabolitesBroccoli}, in der die Brokkolifrucht analysiert wurde, konnte ich etwas mehr davon finden. Tabelle \ref{tab:ComparisonChlKatabolites} listet die jeweiligen Chl-Kataboliten auf:

\begin{table*}[!htbp]\centering
  \ra{1.3}
  
  \begin{tabular}{ccccccccc}\toprule
 Brokkoliblatt & Brokkolifrucht \\
\midrule
\rowcolor{black!20} Bo-NCC-1 & \checkmark \\
 x & Bo-NCC-2 \\ 
\rowcolor{black!20} Bo-NCC-3 & x \\
 Bo-DNCC & \checkmark \\ 
\rowcolor{black!20} Bo-DYCC & x \\
 Bo-YCC & x \\ 
\rowcolor{black!20} Bo-DNCC-2 & x \\
\bottomrule
  \end{tabular}
  
  \caption[Vergleich der Chl-Kataboliten - Brokkoliblatt und Brokkolifrucht, Quelle: Autor]{Vergleich der Chl-Kataboliten im Brokkoliblatt und der Brokkolifrucht}
  \label{tab:ComparisonChlKatabolites}
\end{table*}

\section{Rück- und Ausblick}

Ich denke, mir ist es gelungen, die in der Themenstellung gestellten Fragen weitestgehend zu beantworten. Was mich besonders freut, ist, dass ich mit MS Leafspray eine Methode verwenden sowie weiterentwickeln konnte, die meines Erachtens zusammen mit den Fragmentierungsdiagrammen ein enormes Zukunftspotential hat. Natürlich sind meine Erkenntnisse auf dem Gebiet der Fragmentierungsdiagramme noch nicht gefestigt, doch denke ich, dass ich mit dieser Arbeit einen wesentlichen Grundstein für weitere Forschung in diesem Gebiet legen konnte. 

Durch die parallele Analyse mit LC-MS konnte ein erfolgreicher Vergleich der beiden Methoden gemacht werden, was aufgrund der Bewährtheit von LC-MS ein gutes Licht auf MS Leafspray wirft.

Außerdem konnten andere Chl-Kataboliten im Brokkoliblatt nachgewiesen werden wie in der Frucht an sich, was neue Fragen zum Verständnis des Chl-Abbauprozesses aufwirft.

Demnach leiste ich mit dieser Arbeit einen wesentlichen Beitrag zur Methodenverbesserung, Analyseerweiterung und dem Verständnis des Abbauweges des Chlorophylls im Allgemeinen. Für mich persönlich gilt, dass ich durch diese Arbeit meine Persönlichkeit wesentlich weiterentwickeln konnte, insbesondere was eigenständiges Arbeiten in jeglicher Hinsicht und Sammeln von Erfahrung anbelangt. 

%Ich möchte betonen, dass die Erstellung dieser VWA einiges an Arbeit und vor allem Zeit beanspruchte. Zum Einen war es während meinem einmonatigem Praktikum schwer, sich auf einige wenige Sachen zu konzentrieren, um sinnvoll verwertbare Ergebnisse zu bekommen. Außerdem war das Planen der täglichen Experimente eine Herausforderung für sich. Doch mit der Zeit konnte ich mich in meiner Arbeitsweise zunehmend verbessern und erreichte somit eine große Effizienz\footnote{zu Bestzeiten gelangen mir 4 HPLC Läufe an einem Arbeitstag - das entsprich 5h}. In dem mir zur Verfügung stehendem, kurzem Zeitraum von 1 Monat war es nicht immer leicht, einen Mittelweg zwischen Analyse der Ergebnisse und Durchführen von Experimenten zu finden. Das Resultat war, dass der Großteil der Datenauswertung nach meinem Praktikum erfolgte.