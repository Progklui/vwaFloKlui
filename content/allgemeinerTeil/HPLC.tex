\section{HPLC}

Die \textit{\gls{hplc}} ist eine chromatograpische Methode, um lösliche Stoffe präparativ zu trennen. Es sind dabei quantitative als auch qualitative Analysen möglich. \cite[S. 165]{Chromatographie} \\

Das Trennen der Stoffe basiert auf ihren unterschiedlichen chemischen Eigenschaften. Die Stoffe werden gelöst und bilden zusammen mit dem  \gls{lm} die mobile Phase, die an der stationären Phase vorbeiströmt, wobei es dabei zu Wechselwirkungen zwischen den gelösten Stoffen mit der stationären Phase kommt. Aufgrund der unterschiedlichen chemischen Eigenschaften und den daraus resultierenden unterschiedlichen Wechselwirkungen hält sich jeder Stoff verschieden lange in der stationären Phase auf. Die Verweildauer eines Stoffes in der Trennsäule wird als Retentionszeit bezeichnet. \cite[S. 31-32]{Chromatographie} Die Retentionszeit wird über Detektoren bestimmt, die die Änderung der Zusammensetzung der mobilen Phase feststellen und das Ergebnis in einem Chromatogramm darstellen.  \cite[S. 46]{Chromatographie} \\

Für die Experimente wurde die Methode der \gls{rp} Chromatographie angewandt. Dabei ist die mobile Phase polar und die stationäre Phase unpolar (als unpolare Phase dienen beispielsweise Silane mit langen, unpolaren Kohlenwasserstoffketten). \cite[S. 189]{Chromatographie}\\

Im Rahmen meiner Vorwissenschaftlichen Arbeit wurde die \gls{hplc} verwendet, um die Stoffe im Blatt zu trennen und entsprechende Chlorphyllkataboliten zu extrahieren. Das Identifizieren der Kataboliten erfolgte dabei durch einen UV/VIS Detektor (nahm UV/VIS Spektren im Wellenlängenbereich von 200nm-800nm auf) sowie durch ein dazu geschaltetes Massenspektrometer (=\gls{lcms}). Um die \gls{Chl-K} mit einem hochauflösenden Massenspektrometer zu fragmentieren, wurde der Stoff zu jenen Zeiten, zu denen er in der \gls{hplc} eluiert in Eppendorf Reaktionsgefäßen gesammelt.\\

Die Herstellung eines Blattextraktes für die Analyse mit der \gls{hplc} herzustellen wird in Kapitel \ref{sec:HPLCAufarbeitunderProbe} beschrieben. 
