\chapter{Themenstellung}

Das Ziel der vorliegenden Arbeit besteht darin, die Chlorophyllkataboliten von Brokkoliblättern einer direkten massenspektrometrischen Analyse zu unterziehen und sie anhand unterschiedlicher Merkmale zu untersuchen.

Eine direkte Analyse von Chlorophyllkataboliten umgeht die aufwendige Herstellung eines Blattextraktes, das für eine Analyse mit einer \gls{hplc} sowie durch \gls{lcms} benötigt wird und erlaubt es, Chlorophyllkataboliten ohne vorherige spezielle Aufarbeitung mithilfe eines Massenspektrometers zu analysieren. \cite{DirectPlantTissue} Um eine solche direkte Analyse durchzuführen wurde die Methode des MS-Leafspray \cite{LeafSpray} verwendet (siehe \ref{sec:MSLeafspray} \nameref{sec:MSLeafspray}). \\

Es sollte dabei untersucht werden, inwieweit eine direkte Analyse mittels MS Leafspray eine strukturelle Aufklärung der Chlorophyllkataboliten ermöglicht, sowie, ob das Stattfinden einer Reaktion von Essigsäureanhydrid mit den Chlorophyllkataboliten festgestellt werden kann. Die Reaktion selbst sollte Aufschluss über die Reaktivitäten unterschiedlicher Carbonsäuren der Chlorophyllkataboliten und deren struktureller Besonderheiten geben. \\

Um die Ergebnisse zu überprüfen wurden eine \gls{hplc} sowie ein hochauflösendes Massenspektrometer verwendet, das unter anderem an die \gls{hplc} gekoppelt war. Diese Methoden setzen die Herstellung eines Blattextraktes voraus und fallen somit nicht mehr unter die direkte Analyse. Es wird somit auch ein Vergleich dieser beiden Methoden angestrebt. 

Was den Vergleich der beiden Methoden anbetrifft wurden sowohl beim hochauflösenden Massenspektrometer als auch mit MS-Leafspray Fragmentierungsdiagramme der Kataboliten im \gls{cid} Modus erzeugt, mit dem Ziel, eventuelle charakteristische Eigenschaften herauszufinden und zu vergleichen, inwieweit Fragmentierungen des hochauflösenden Massenspektrometers mit MS-Leafspray reproduzierbar sind. 




