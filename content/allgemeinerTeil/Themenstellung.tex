\chapter{Themenstellung}

Das Ziel der vorliegenden Arbeit besteht darin, die \gls{Chl-K}en von Brokkoliblättern einer direkten massenspektrometrischen Analyse zu unterziehen und sie anhand unterschiedlicher Merkmale zu untersuchen.

Eine direkte Analyse von \gls{Chl-K}en umgeht die aufwendige Herstellung eines Blattextraktes, das für eine Analyse mit einer \gls{hplc} sowie durch \gls{lcms} benötigt wird und erlaubt es, \gls{Chl-K}en ohne vorherige spezielle Aufarbeitung mithilfe eines Massenspektrometers \textit{direkt} zu analysieren. \cite{DirectPlantTissue} Für die Durchführung einer solchen direkten Analyse wurde die Methode des MS-Leafspray \cite{LeafSpray} verwendet (siehe \ref{sec:MSLeafspray} \nameref{sec:MSLeafspray}). \\

Es sollte dabei untersucht werden, inwieweit eine direkte Analyse mittels MS Leafspray eine strukturelle Aufklärung von \gls{Chl-K}en ermöglicht, sowie, ob das Stattfinden einer Reaktion von Essigsäureanhydrid mit den \gls{Chl-K}en festgestellt werden kann. Die Reaktion selbst sollte Aufschluss über die Reaktivitäten unterschiedlicher Carbonsäuren der \gls{Chl-K}en und deren struktureller Besonderheiten geben. \\

Um die Ergebnisse zu überprüfen wurden eine \gls{hplc} sowie ein hochauflösendes Massenspektrometer verwendet. Eine Analyse mit diesen Methoden setzt die Herstellung eines Blattextraktes voraus und fällt deswegen nicht mehr unter die direkte Analyse. Es wird damit ein Vergleich dieser beiden Methoden angestrebt. 

Was den Vergleich der beiden massenspektrometrischen Methoden anbetrifft wurden sowohl beim hochauflösenden Massenspektrometer als auch mit MS-Leafspray Fragmentierungsdiagramme der Kataboliten im \gls{cid} Modus erzeugt, mit dem Ziel, eventuelle charakteristische Eigenschaften herauszufinden und zu vergleichen, inwieweit Fragmentierungen sowie Fragmentierungsdiagramme des hochauflösenden Massenspektrometers mit MS-Leafspray reproduzierbar sind. Mit den Fragmentierungsdiagrammen wird zudem versucht, eine zukünftig bessere Strukturaufklärung zu ermöglichen.




