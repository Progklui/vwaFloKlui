\chapter{Themenstellung}

Das Ziel der vorliegenden Arbeit besteht darin, die \gls{Chl-K} von Brokkoliblättern einer direkten massenspektrometrischen Analyse zu unterziehen und sie anhand unterschiedlicher Merkmale zu untersuchen.

Es sollte dabei untersucht werden, inwieweit eine direkte Analyse mit der noch relativ neuen Methode des MS Leafspray eine strukturelle Aufklärung von \gls{Chl-K}en ermöglicht, sowie, ob das Stattfinden einer Reaktion von Essigsäureanhydrid mit den \gls{Chl-K}en festgestellt werden kann. 

Die Reaktion selbst sollte Aufschluss über die Reaktivitäten unterschiedlicher Carbonsäuren der \gls{Chl-K}en und deren struktureller Besonderheiten geben. \\

Mithilfe von im \gls{cid} Modus erstellten Fragmentierungsdiagrammen wird versucht, eventuelle charakteristische Eigenschaften der \gls{Chl-K}en ausfindig zu machen. Diese Eigenschaften können sich \gls{zB} im Vergleich der aufgenommenen Diagramme zwischen den \gls{Chl-K}en ergeben aber auch im intramolekularen Bereich. 

Es soll somit das Potential von Fragmentierungsdiagrammen in Hinblick auf zukünftige Strukturaufklärung mithilfe eines Massenspektrometers analysiert werden. \\

Um die Ergebnisse von MS Leafspray zu überprüfen wurde die Methode der \gls{hplc} sowie ein hochauflösendes Massenspektrometer verwendet. Mit den im hochauflösenden Massenspektrometer erhaltenen Fragmentierungen wird versucht, diese mit den von MS Leafspray erhaltenen zu vergleichen, um einen Vergleich der beiden Methoden abgeben zu können.

Eine Analyse mit diesen Methoden setzt die Herstellung eines Blattextraktes voraus und verdient deswegen nicht mehr die Bezeichnung der direkten Analyse.




