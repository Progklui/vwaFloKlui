\chapter{Themenstellung} \label{sec:Themenstellung}

Das Ziel der vorliegenden Arbeit besteht darin, die \gls{Chl-K} von Brokkoliblättern einer direkten massenspektrometrischen Analyse zu unterziehen und sie anhand unterschiedlicher Merkmale zu untersuchen.

Es sollte dabei untersucht werden, inwieweit eine direkte Analyse mit der modernen Methode MS-Leafspray eine strukturelle Aufklärung von \gls{Chl-K}en ermöglicht sowie, ob das Stattfinden einer Reaktion von Essigsäureanhydrid mit den \gls{Chl-K}en festgestellt werden kann. 

Als Ergebnisse der Reaktion werden Anhydride als Reaktionsprodukte sowie ein Verständnis über die Reaktivitäten von Carbonsäuren der \gls{Chl-K}en und deren struktureller Besonderheiten erwartet.\\

Mithilfe von im \gls{cid} Modus erstellten Fragmentierungsdiagrammen wird versucht, charakteristische Eigenschaften der \gls{Chl-K}en ausfindig zu machen. Diese Eigenschaften können sich z. B. im Vergleich der aufgenommenen Diagramme zwischen den \gls{Chl-K}en oder im intramolekularen Bereich ergeben. 

Es soll somit das Potential von Fragmentierungsdiagrammen mit Hinblick auf zukünftige Strukturaufklärung mithilfe eines Massenspektrometers analysiert werden. \\

Um die Ergebnisse von MS-Leafspray zu überprüfen wurde die Methode der \gls{hplc} sowie ein hochauflösendes Massenspektrometer verwendet. Mit den im hochauflösenden Massenspektrometer erhaltenen Fragmentierungen wird versucht, diese mit jenen von MS-Leafspray zu vergleichen, um einen Vergleich der beiden Methoden anstellen zu können.





