\section{Massenspektrometrie}

Mithilfe eines Massenspektrometers kann die Masse eines Moleküls bestimmt werden. Aufgrund der Einfachheit der Methode und der sehr geringen benötigten Probenmenge ist das Massenspektrometer für eine Vielzahl an Anwendungen geeignet (z. B. in der Forensik, Medikamentenprüfung, Analyse von Meteoriten). \cite[S. 1]{MassSpectrometry} \\

Um die Molekülmasse der Stoffe zu bestimmen, werden sie zuerst in Gasphasen-Ionen überführt. \cite[S. 15]{MassSpectrometry} Dabei gibt es unterschiedliche Methoden, diesen Zustand herbeizuführen, wie z. B. Elektronenstoß-Ionisation, Chemische-Ionisation und Feld-Ionisation. \cite[S. 15-30]{MassSpectrometry} Nach der Ionisation werden sie im Massenanalysator nach ihrem \gls{mz} Verhältnis getrennt und im Detektor der Ionenstrom gemessen. Das Ergebnis wird in einem Massenspektrum festgehalten, in dem auf der Ordinate die relative Intensität der einzelnen Peaks und auf der Absizze das Verhältnis \gls{mz} aufgetragen werden. \\

In den Experimenten dieser Arbeit wurde zur Ionisation die \gls{ESI} Methode verwendet, die erstmalig das Messen von Proteinen mithilfe eines Massenspektrometers erlaubte und aufgrund ihrer hohen Empfindlichkeit gegenüber kleinen, polaren Molekülen mit einer \gls{hplc} kombiniert werden kann. Bei der \gls{ESI} Methode wird durch Anlegen einer Spannung von 3-6 kV zwischen der Kapillare, aus der die Flüssigkeit kommt und der Gegenelektrode ein elektrisches Feld mit einer Stärke in der Größenordnung von $10^{6}$ \si{\V\per\m} angelegt. Die erhaltenen geladenen Tröpfchen passieren ein Inertgas (in den meisten Fällen Stickstoff) bzw. eine erhitzte Kapillare, um das \gls{lm} zu entfernen. Anschließend an diese Ionisation wird die Molekülmasse der Ionen bestimmt. \cite[S. 43-44]{MassSpectrometry} \\

Um die Chlorophyllkataboliten im Massenspektrometer zu analysieren, wurde sowohl die Methode der \gls{lcms} als auch die Methode MS-Leafspray verwendet. 

\subsection{LC-MS}

Bei der Methode der LC-MS wird eine \gls{hplc} mit einem Massenspektrometer gekoppelt. Dabei trennt die \gls{hplc} die Stoffe zuvor auf und eluiert sie anschließend in das Massenspektrometer. \cite[S. 217-218]{MassSpectrometry} Um die Flussrate bei atmosphärischem Druck zu verringern, wird nur ein Teil des direkt aus der \gls{hplc} kommenden Flusses zum Massenspektrometer geleitet. Ansonsten wäre die Flussrate zu hoch, was eine Ionisierung der Probe mithilfe einer \gls{ESI} - Quelle unmöglich machen würde. \cite[S. 221]{MassSpectrometry} 

Das Resultat ist je ein Chromatogramm vom HPLC- und Massenspektrometerlauf. Es wird somit zu jedem Zeitpunkt eines \gls{hplc}-Laufes ein UV/Vis Spektrum sowie ein Massenspektrum erzeugt. Aus dem UV/Vis Spektrum lässt sich schließen, ob es sich bei einem \gls{Chl-K}en um einen \gls{NCC}, \gls{DNCC} oder einen \gls{YCC} handelt. Aus dem Massenspektrum kann die Molekülmasse (in Da) mit allgegenwärtigen Fragmentierungen abgelesen werden. Unter Verwendung eines hochauflösenden Massenspektrometers wird außerdem die elementare Zusammensetzung ersichtlich. \\

\subsection{MS-Leafspray} \label{sec:MSLeafsprayTheoretisch}

\textit{Ambient Ionization} \cite{AmbientIonisation} ermöglicht es, Proben ohne vorherige analytische Trennung durch chromatographische Trennverfahren direkt in ihrer \textit{natürlichen} Umgebung mithilfe eines Massenspektrometers zu untersuchen. Eine Methode, die auf dem Prinzip der \textit{Ambient Ionization} basiert ist \textit{Paper Spray} \cite{PaperSpray}. Dabei  kommt es zu einer Kombination der \gls{ESI} sowie der \textit{Ambient} Ionisationsmethode. \cite{PaperSpray}\\

Die Ionisation der Probe erfolgt ausgehend von einem feuchten, porösen Material (z. B. Papier), das in einer Kupferklemme eingeklemmt wird. Zwischen der Kapillaröffnung des Massenspektrometers und der Kupferklemme liegt eine Spannung im Bereich von 3-6 kV an, woraufhin kleine Tröpfchen, die Ionen der Probe enthalten, von der Spitze des porösen Materials ausgesendet werden und Ionen der Probe in das Massenspektrometer befördern. \cite{RapidScreeningLeafSpray} Durch Anlegen von Kalibrationskurven mit externen Standards wird außerdem ermöglicht, eine quantitative Bestimmung der Menge des Analyten durchzuführen. \cite{LeafSpray}

Leaf Spray ist eine Form von Paper Spray, bei der die zu analysierende Pflanze selbst als poröses Material dient. Sie wurde im Rahmen dieser Arbeit für die Identifikation von \gls{Chl-K}en verwendet. Ein Vorteil einer Analyse von \gls{Chl-K}en mit MS-Leafspray ist, dass weniger Zeit für die Vorbereitung benötigt wird, was wiederum Grundlage für eine schnellere und effizientere Analyse ist. Genauere Ausführungen zur Durchführung und eine Auflistung weiterer Vor- und Nachteile finden sich in Kapitel \ref{sec:MSLeafspray} und \ref{sec:VergleichDirektKlassisch}.\\