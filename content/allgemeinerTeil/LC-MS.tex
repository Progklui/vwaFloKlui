\section{Massenspektrometrie}

Mithilfe eines Massenspektrometers kann die Masse eines Moleküls bestimmt werden. Aufgrund der Einfachheit der Methode und der sehr geringen benötigten Probenmenge ist das Massenspektrometer für eine Vielzahl an Anwendungen geeignet (\gls{zB} in der Forensik, Lebensmittelprüfung, Medikamentenprüfung, ...). \cite[S. 1]{MassSpectrometry} Der jetzige Entwicklungsstand in der Massenspektrometrie ist vor allem den Entwicklungen in den letzten vier Jahrzehnten auf diesem Gebiet zu verdanken. \cite[S. 6-9]{MassSpectrometry} \\

Um die Molekülmasse der Stoffe zu bestimmen, werden sie zuerst in Gasphasen-Ionen überführt. \cite[S. 15]{MassSpectrometry} Dabei gibt es unterschiedliche Methoden, diesen Zustand herbeizuführen, wie \gls{zB} \gls{EI}, \gls{CI} und \gls{FI}. \cite[S. 15-30]{MassSpectrometry} Nach der Ionisation können die Stoffe detektiert werden. Das Ergebnis wird in einem Massenspektrum festgehalten, in dem auf der Ordinate die relative Intensität der einzelnen Peaks und auf der Absizze das Verhältnis \gls{mz} aufgetragen werden, da der Detektor technisch gesehen nicht die Molekülmasse misst, sondern nur obiges Verhältnis. \\

In den Experimenten dieser Arbeit wurde zur Ionisation die \gls{ESI} Methode verwendet, die erstmalig das Messen von Proteinen mithilfe eines Massenspektrometers erlaubte und aufgrund ihrer hohen Empfindlichkeit gegenüber kleinen, polaren Molekülen mit einer \gls{hplc} kombiniert werden kann. Dabei wird durch Anlegen einer Spannung von 3-6kV zwischen der Kapillare, aus der die Flüssigkeit kommt und der Gegenelektrode ein elektrisches Feld mit einer Stärke in der Größenordnung von $10^{6}$ $Vm^{-1}$ angelegt. Die erhaltenen geladenen Tropfen passieren dann ein Inertgas (in den meisten Fällen \gls{n2}) \gls{bzw} eine erhitzte Kapillare, um das \gls{lm} zu entfernen. \cite[S. 43-44]{MassSpectrometry} \\

Um die Chlorophyllkataboliten im Massenspektrometer zu analysieren wurde sowohl die Methode der \gls{lcms} als auch die Methode des MS Leafspray verwendet. 

\subsection{LC-MS}

Bei der Methode der LC-MS wird eine \gls{hplc} vor ein Massenspektrometer geschaltet. Dabei trennt die \gls{hplc} die Stoffe zuvor auf und eluiert sie anschließend in das Massenspektrometer. \cite[S. 217-218]{MassSpectrometry} Das Resultat ist je ein Chromatogramm der HPLC und des Massenspektrometers. Es wird somit zu jedem Zeitpunkt eines \gls{hplc}-Laufes ein UV/VIS sowie ein Massenspektrum erzeugt. Aus dem UV/VIS Diagramm lässt sich schließen, ob es sich bei einem Kataboliten um einen \gls{NCC}, \gls{DNCC} oder einen \gls{YCC} handelt (finden von Referenz, in denen die einzelnen Banden \textit{entdeckt} wurden). Aus dem Massenspektrum wird die Molekülmasse [M+H]\textsuperscript{+} (in Da) ersichtlich. Unter Verwendung eines hochauflösenden Massenspektrometers wird außerdem die atomare Zusammensetzung ersichtlich. \\

Um die Flussrate bei atmosphärischem Druck zu verringern, wird nur ein Teil des direkt aus der \gls{hplc} kommenden Flusses zum Massenspektrometer hin abgezweigt. Ansonsten wäre die Flussrate zu hoch, was eine Ionisierung der Probe mithilfe einer \gls{ESI}-Quelle unmöglich machen würde. \cite[S. 221]{MassSpectrometry}


\subsection{MS Leafspray} \label{sec:MSLeafspray}

\textit{Ambient Ionization} \cite{AmbientIonisation} ermöglicht es, Proben ohne vorherige präparative Trennung durch chromatographische Trennverfahren direkt in ihrer \textit{natürlichen} Umgebung mithilfe eines Massenspektrometers zu untersuchen. Eine Methode, die auf dem Prinzip der \textit{Ambient Ionization} basiert ist \textit{Paper Spray} \cite{PaperSpray}. Dabei  kommt es zu einer Kombination der \gls{ESI} sowie der \textit{Ambient} Ionisationsmethode \cite{PaperSpray}. 

Die Ionisation der Probe erfolgt ausgehend von einem feuchten, porösen Material (\gls{zB} Papier), das zwischen eine Kupferklemme geklemmt wird. Zwischen der Kapillaröffnung des Massenspektrometers und der Kupferklemme liegt eine Spannung im Bereich von 3-6kV an, woraufhin kleine Tröpfchen, die Ionen der Probe enthalten von der Spitze des porösen Materials ausgesendet werden und Ionen der Probe in das Massenspektrometer befördern. \cite{RapidScreeningLeafSpray} Durch Anlegen von Kalibrationskurven mit externen Standards wird außerdem ermöglicht, eine quantitative Bestimmung der Menge des Analyten durchzuführen. \cite{LeafSpray}

Leaf Spray ist eine Form von Paper Spray, bei der die zu analysierende Pflanze selbst als poröses Material dient. 

