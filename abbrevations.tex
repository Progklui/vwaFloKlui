% Allgemeine Abkürzungen
\newacronym{lm}{LM}{Lösungsmittel}
\newacronym{meoh}{MeOH}{Methanol}
\newacronym{eppi}{EPPI}{Eppendorf Reaktionsgefäß}
\newacronym{zB}{z.B.}{zum Beispiel}
\newacronym{bzw}{bzw.}{beziehungsweise}
\newacronym{uA}{u.a.}{unter anderem}
\newacronym{ca}{ca.}{circa}
\newacronym{nAb}{nAb.}{noch Aufklärungsbedarf}

% Chromatographie Abkürzungen
\newacronym{hplc}{HPLC}{High performance liquid chromatography}
\newacronym{rp}{RP}{Reversed-phase}
\newacronym{lcms}{LC-MS}{Liquid Chromatography-Mass Spectrometry}

% Massenspektrometrie Abkürzungen
\newacronym{EI}{EI}{Electron Ionization}
\newacronym{ESI}{ESI}{Electrosprayionisation}
\newacronym{CI}{CI}{Chemical Ionization}
\newacronym{FI}{FI}{Field Ionization}
\newacronym{mz}{m/z}{Masse pro Ladung}
\newacronym{cid}{CID}{Collision induced Dissociation}
\newacronym{pqd}{PQD}{PQD}
\newacronym{nKE}{NKE}{normalisierten Kollisionsenergie (in \%)}

% Chlorophyllkataboliten Abkürzungen
\newacronym{NCC}{NCC}{Non flourescent Chlorophyllic Catabolite}
\newacronym{YCC}{YCC}{Flourescent Chlorophyllic Catabolite}
\newacronym{DNCC}{DNCC}{Decarboxylated-Non flourescent Chlorophyllic Catabolite}
\newacronym{Chl-K}{Chl-Katabolit}{Chlorophyll Katabolit}

% chemische Formeln Abkürzungen
\newacronym{n2}{N2}{Stickstoff}