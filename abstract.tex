%%%% Time-stamp: <2013-02-25 10:31:01 vk>

\pagenumbering{gobble}

\chapter*{Abstract}
\label{cha:abstract}

Chlorophyllkataboliten sind die Endprodukte des Abbauprozesses von Chlorophyll. Im Rahmen eines Praktikums am Organischen Institut der Universität Innsbruck wurden die Chl-Kataboliten frischer Brokkoliblätter einer direkten Analyse mit MS Leafspray unterzogen. 

MS Leafspray stellt dabei eine neue Methode der Massenspektrometrie dar, die es ermöglicht, Probenmaterial in natürlicher Umgebung zu analysieren. Nach einer Erstidentifikation über MS Leafspray wurde das Ergebnis mit LC-MS verifiziert. Die mit beiden Methoden gefundenen Chl-Kataboliten lauten wie folgt: Bo-NCC-1, Bo-NCC-3, Bo-DNCC, Bo-DNCC-2, Bo-DYCC und Bo-YCC. Im Vergleich zum Brokkoliblatt konnten 4 weitere Chl-Kataboliten gefunden werden, was neue Fragen in Bezug auf das Verständnis des Abbauprozesses aufwirft. Für jeden der genannten Chl-Kataboliten konnten Strukturvorschläge gemacht werden, die noch mit \textsuperscript{1}H-NMR überprüft werden müssten.

Außerdem wurde eine Reaktion am Blatt der Chl-Kataboliten mit Essigsäureanhydrid durchgeführt. Die Reaktionsprodukte (Anhydride) konnten mit MS Leafspray durch die Beobachtung einer Massenzunahme nachgewiesen werden. Auf Basis diverser Fragmentierungen wird vorgeschlagen, dass die Reaktion nur an einer der beiden freien Carbonsäuren der Chl-Kataboliten erfolgt. 

Im Rahmen der massenspektrometrischen Analysen wurden Fragmentierungsdiagramme erstellt, von denen angesichts der Ergebnisse vermutet wird, dass sie charakteristisch für bestimmte Chl-Kataboliten sind. Eine Interpretationsmöglichkeit der Diagramme konnte vorgeschlagen werden. 

Es wurde somit ein wesentlicher Beitrag zur Weiterentwicklung der direkten Analyse mit MS Leafspray geleistet sowie konnte diese Methode mit der klassischen LC-MS verglichen werden. Zudem wurde ein Grundstein für die weitere Erforschung von Fragmentierungsdiagrammen und ihrer Aussagekraft gelegt. Zu guter Letzt konnte einiges zum Verständnis der Durchführung einer Reaktion mit Essigsäureanhydrid beigetragen werden.



%\glsresetall %% all glossary entries should be used in long form (again)
%% vim:foldmethod=expr
%% vim:fde=getline(v\:lnum)=~'^%%%%\ .\\+'?'>1'\:'='
%%% Local Variables:
%%% mode: latex
%%% mode: auto-fill
%%% mode: flyspell
%%% eval: (ispell-change-dictionary "en_US")
%%% TeX-master: "main"
%%% End:
